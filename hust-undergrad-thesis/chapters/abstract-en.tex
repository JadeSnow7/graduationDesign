\chapter*{Abstract}
\addcontentsline{toc}{chapter}{Abstract}
This thesis addresses common challenges in higher education, including diverse course types, significant learner differences, the high cost of process-oriented feedback, and the difficulty of guaranteeing reliability in generative models. We design and implement a student-centric intelligent teaching platform powered by large language models (LLMs). The platform organizes learning-process data (submissions, tutoring dialogues, and learning events) into longitudinal student profiles, enabling controllable, traceable, and iterative tutoring. The system follows a service-oriented architecture with a separated front end and back end: the client is built with React, TypeScript, and Vite for web/H5 (embeddable in a WeCom WebView); the backend uses Go and Gin to provide business APIs with JWT-based authentication and RBAC; and the AI service is implemented with Python and FastAPI to orchestrate chat, writing analysis, guided learning, tool calling, and retrieval augmentation, connecting to upstream inference via an OpenAI-compatible API. The heterogeneous acceleration layer supports GPU, NPU (Huawei Ascend), and FPGA (Xilinx Alveo) backends: GPU/NPU for LLM inference, and FPGA for low-latency embedding acceleration and resource decoupling, significantly reducing power consumption while maintaining retrieval quality. To reduce hallucinations and improve auditability, we introduce GraphRAG: course materials are indexed as a graph-structured knowledge base, and responses are grounded on retrieved evidence with explicit citations. For verifiable tasks such as numerical calculation/simulation or format checks, tool calling delegates key steps to executable tools and injects results back into the dialogue. To support model customization and continuous iteration, we provide a data specification, LoRA/QLoRA fine-tuning and offline evaluation scripts, as well as data distillation and smoke validation for pipeline sanity checks. Two representative scenarios---an electromagnetics derivation-intensive course and a graduate academic writing course---are used to demonstrate portability and engineering feasibility.

\vspace{0.5cm}
\noindent\textbf{Keywords:} intelligent teaching platform; student profile; GraphRAG; tool calling; guided learning; heterogeneous acceleration; FPGA
\clearpage
