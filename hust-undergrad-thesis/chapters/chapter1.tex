\chapter{绪论}
\section{研究背景与意义}
高校课程呈现出显著的多样性:一类是推导与计算密集的理工科课程(例如本科生《电磁场与电磁波》),其学习难点在于概念依赖关系复杂、公式推导链条长且对计算正确性要求极高;另一类是过程性与评价维度复杂的技能型课程(例如研究生《学术规范与论文写作》),其学习难点在于论证组织、学术语气、证据与引用规范等能力需要长期迭代训练。两类课程在教学组织上都面临共同痛点:学生个体差异显著、过程性学习数据难以系统沉淀、教师在答疑与反馈中的边际成本高,导致“高频但低复用”的重复劳动难以规模化优化。

近年来,Transformer 架构\cite{vaswani2017}推动大语言模型(LLM)在自然语言理解与生成方面取得突破,对话式模型(如 ChatGPT)\cite{openai2022}为教学辅助提供了新的交互范式:既可解释概念、提供建议,也可进行循序渐进的引导。然而,LLM 在教育场景中面临可靠性与可治理性挑战:模型可能产生缺乏依据的建议(幻觉),对推导/格式等“硬约束”任务容易出现不可复核的错误,且难以稳定贴合不同课程的评价标准。检索增强生成(RAG)\cite{lewis2020}与知识图谱\cite{hogan2020}为上述问题提供了可追溯的外部知识约束:通过将课程规范、讲义、示例与题解片段构建为可检索证据库,让生成输出有据可依、可引用溯源。在此基础上,引入 GraphRAG 的“片段检索 + 图扩展”机制,可进一步利用知识间的结构化关系提升关联检索与多跳推理能力。与此同时,工具调用(Function Calling)把关键计算/校验步骤交由可执行工具完成,为数值计算、仿真或写作结构/引用规范检查等任务提供可验证路径。

基于企业微信内嵌 H5 或网页的入口具备便捷触达与组织管理优势,适合承载可扩展的智能教学平台原型。本文以本科生《电磁场与电磁波》与研究生《学术规范与论文写作》作为示例场景,探索一种强调“学生中心、可追溯、可验证、可迭代”的智能教学平台设计与实现方法。

\section{国内外研究现状}
在教育信息化领域,学习管理系统(LMS)与学习分析工具已能支持课程资源管理、作业提交与学习过程记录,但对“解释 + 引导 + 过程性反馈”的深层教学任务支持有限。近年 LLM 在教育问答、自动反馈与写作辅助等场景中应用快速增长,研究重点逐步从“能生成”转向“可追溯、可验证、可治理”:RAG 被广泛用于降低幻觉并提升答案依据\cite{lewis2020};知识图谱在教材知识组织、概念关联与检索推荐中具有优势\cite{hogan2020},但与 LLM 深度协同、并能同时服务多课程场景的系统化平台仍处于快速演进阶段。

在检索增强生成方面,传统 RAG 通过向量检索注入外部知识以降低幻觉,但在知识关联性强、需要多跳依赖的任务中仍存在召回不足与上下文碎片化等问题。GraphRAG 将文档组织为“节点—边—片段”的图结构索引,支持从种子片段出发的图扩展,以更好地覆盖前置概念、相关规范与示例。对于需要“硬约束”的任务,工具调用机制通过结构化参数把计算/校验委托给可执行工具,减少模型自由生成导致的错误。综上,构建面向多课程的智能教学平台,并将检索、工具与训练能力工程化落地,具有明确的研究与应用价值。

\section{研究问题与创新目标}

\subsection{核心研究问题}
本文围绕以下四个核心研究问题展开:
\begin{enumerate}[label=(RQ\arabic*)]
  \item 如何在不同课程任务中降低幻觉,并保证关键结论具备可验证性(例如数值计算/仿真、格式与结构检查)?
  \item 如何构建可追溯的知识检索机制,既支持多跳关联检索,也支持证据引用与来源标注?
  \item 如何把“教学风格与引导策略”显式化,使模型输出在不同课程中可控且可迁移?
  \item 如何将作业、讨论、讲义与规范等多源数据纳入统一数据闭环,并支撑持续迭代与回归评测?
\end{enumerate}

\subsection{创新目标}
针对上述研究问题,本文提出以下可验证的创新目标:
\begin{enumerate}[label=(\arabic*)]
  \item 提出并落地“学生中心数据闭环 + GraphRAG 可追溯 + 工具调用可验证 + 引导式学习”的一体化框架,形成可复用的工程实现路径。
  \item 设计“通用能力 + 课程专属模块”的课程模块化策略,以《电磁场与电磁波》与《学术规范与论文写作》为例验证跨课程可迁移性与能力边界。
  \item 建立面向持续迭代的训练与回归评测流程:数据规范化、数据蒸馏与 smoke 自检、LoRA/QLoRA 训练与离线评测报告输出。
\end{enumerate}

\section{研究内容与任务要求}
本文围绕面向多课程的智能教学平台设计与实现开展研究,主要内容与任务要求如下。

\subsection{课题内容}
\begin{enumerate}[label=(\arabic*)]
  \item 平台总体架构与工程化实现:React 前端、Go 后端与 FastAPI AI 服务的分层与接口契约设计,支持企业微信内嵌 H5/网页访问。
  \item GraphRAG 知识库构建与检索增强生成:面向课程材料/规范/示例构建图结构索引,并实现引用编号的可追溯回答。
  \item 工具调用与可验证执行器:对数值计算/仿真、写作结构与引用规范检查等任务提供工具调用能力与结果回注机制。
  \item 引导式学习与学习画像:生成学习路径并逐步引导,记录薄弱点与学习进度,支撑过程性反馈闭环。
  \item 模型后训练与评测管线:数据规范、数据蒸馏与 smoke 验证、LoRA/QLoRA 微调与离线评测报告输出。
  \item 示例场景验证与测试评估:以电磁场与研究生专业英文写作为例进行端到端演示与评估设计。
\end{enumerate}

\subsection{课题任务要求}
\begin{enumerate}[label=(\arabic*)]
  \item 深入理解大语言模型的基本原理及其应用范式,掌握至少一种主流 LLM 的 API 调用方法。
  \item 掌握知识图谱与 GraphRAG 的构建流程,能够将课程讲义、规范与示例构建为可检索的证据库,并在回答中实现引用溯源。
  \item 实现工具调用机制,支持数值计算/仿真与规则化检查任务的可验证执行与结果回注。
  \item 完成一个可部署的系统原型,覆盖鉴权、权限治理、知识库检索增强、对话与引导式学习等关键链路。
  \item 原型系统应体现跨课程可迁移性:在电磁场与研究生专业英文写作等示例任务上能够输出结构化、可追溯、可复核的辅助结果。
  \item 完成毕业设计论文的撰写,论文应结构清晰、论证充分、代码和数据详实。
\end{enumerate}

\section{论文结构}
全文共分六章:第一章为绪论,介绍研究背景意义、国内外研究现状、研究问题与创新目标,并给出研究内容与任务要求;第二章阐述平台涉及的关键技术与理论基础,包括 React 前端、Go 后端、GraphRAG、技能系统、工具调用、引导式学习、学习状态追踪以及 LoRA/QLoRA 与数据蒸馏等;第三章给出系统需求分析与总体设计,明确业务流程、数据与权限模型及跨课程可扩展方案(以电磁场与研究生专业英文写作为例);第四章面向工程落地,详细介绍 Skills/Tool Calling/GraphRAG、引导式学习与学习档案、训练与评测管线等关键模块的设计与实现;第五章对系统进行测试与实验评估,从功能正确性、引用一致性、工具调用可用性与回归评测等方面验证方案效果;第六章总结全文并展望后续工作。
