\chapter{系统需求分析与总体设计}

\section{应用场景与用户角色}
本项目面向高校多课程教学场景,强调以学生为中心的过程性支持与可追溯治理:既要覆盖理工科推导/计算密集任务(以电磁场为例),也要覆盖写作类“过程性训练 + 评价维度复杂”的任务(以研究生专业英文写作为例)。系统用户角色主要包括:学生(学习与提交)、教师/助教(发布与反馈、学情分析)以及管理员(课程与权限配置)。在上述角色协作下,平台需同时满足“日常答疑/辅导”的即时交互需求与“长期学习状态沉淀”的过程性数据需求。

\section{需求分析}

\subsection{功能性需求}
结合平台定位与多课程适配目标,系统核心功能需求如下:
\begin{enumerate}[label=(\arabic*)]
  \item \textbf{对话式辅导与多模式能力切换}:支持概念讲解、作业反馈、引导式学习、个性化辅导策略生成等模式,并可在课程间复用通用能力。
  \item \textbf{可追溯检索增强}:支持将课程讲义、作业规范与示例等构建为证据库,回答需携带引用编号,支持教师复核与溯源审计。
  \item \textbf{可验证执行能力}:对符号推导、数值计算、仿真与写作规则检查等任务,支持工具调用并回注结果,形成可复核的“计算/校验链”。
  \item \textbf{引导式学习与进度推进}:围绕学习主题生成学习路径并分步提问推进,支持会话恢复与步骤状态更新。
  \item \textbf{学习画像与长期状态追踪}:沉淀薄弱点、完成主题、学习时长等信息,形成课程画像与跨课程全局画像,并支持学习时间线查询。
  \item \textbf{训练与评测闭环}:支持数据规范化、数据蒸馏与 smoke 门禁、LoRA/QLoRA 微调与离线回归评测,为能力迭代提供可复现链路。
\end{enumerate}

\subsection{非功能性需求}
为保证系统在真实教学环境中的可用性与可治理性,非功能需求包括:
\begin{enumerate}[label=(\arabic*)]
  \item \textbf{可信与可复核}:关键结论需能以“引用证据 + 工具结果”进行复核,降低幻觉影响。
  \item \textbf{安全与权限}:基于 RBAC 的角色权限控制,支持课程维度的数据隔离与访问控制;工具调用需具备安全约束,避免越权与资源滥用。
  \item \textbf{可扩展与可维护}:能力扩展应以模块化方式接入(新增技能/工具/课程材料),避免改动核心接口引发兼容性风险。
  \item \textbf{性能与稳定性}:在企业微信 WebView/浏览器等多终端环境中保持稳定交互;检索与工具执行需具备超时与降级策略。
  \item \textbf{隐私与学术诚信}:对学生数据与作业内容进行必要保护;对代写等学术不端请求具备拒答/引导策略。
\end{enumerate}

\section{总体架构设计}

\subsection{分层与服务划分}
系统采用前后端分离与服务化架构,整体分为三层:
\begin{enumerate}[label=(\arabic*)]
  \item \textbf{Web/H5 客户端}:基于 React 构建交互界面,负责对话、学习进度展示、写作分析结果呈现与教师侧学情概览等。
  \item \textbf{业务后端}:基于 Go + Gin 提供统一鉴权、RBAC 权限治理、课程与作业管理、学习画像/事件接口等业务能力。
  \item \textbf{AI 服务层}:基于 FastAPI 提供 OpenAI-compatible 的对话接口与可插拔能力,包括 Skills、GraphRAG 检索增强、Tool Calling 执行器以及引导式学习会话管理等。
\end{enumerate}
该分层使“教学业务逻辑”与“模型推理与能力编排”解耦:后端负责权限、数据与流程治理,AI 服务负责生成与可验证能力落地,前端负责交互与可解释呈现。

\subsection{关键业务流程}
以一次“学生提问/辅导”为例,核心链路可概括为:前端提交问题与上下文 $\rightarrow$ 后端鉴权与权限校验 $\rightarrow$ AI 服务选择技能(\texttt{mode})并可选启用 GraphRAG(\texttt{\_rag} 后缀) $\rightarrow$ 若需要可验证结果则触发工具调用 $\rightarrow$ 回注工具结果与引用证据 $\rightarrow$ 生成最终回答并返回 $\rightarrow$ 后端记录学习事件与画像更新。该流程将生成式回答纳入“可追溯(证据)+ 可验证(工具)+ 可沉淀(画像/事件)”的数据闭环。

\section{模块划分与接口约定}
为便于后续章节的实现描述与复现,本节给出平台关键能力与模块化落地的对应关系:
\begin{table}[htbp]
  \centering
  \caption{关键能力与实现模块对应关系}
  \begin{tabular}{p{3.2cm}p{9.2cm}}
    \toprule
    能力模块 & 实现要点(模块化落地) \\
    \midrule
    Skills(提示编排) & 技能注册表统一管理系统提示与约束,\texttt{mode} 选择技能并支持别名映射;技能可注入课程/作业/画像等上下文。 \\
    GraphRAG(可追溯检索) & 基于图结构索引进行“种子检索 + 图扩展”,回答强制引用编号;支持混合检索与索引更新与 ACL 过滤。 \\
    Tool Calling(可验证执行) & 统一工具参数 Schema 与白名单,限制最大调用次数;工具执行结果回注对话上下文用于最终解释。 \\
    Guided Learning(过程性辅导) & 学习路径生成 + 分步提问推进;会话状态(TTL、用户绑定)维护进度与薄弱点。 \\
    训练与评测 & chat-style 数据规范、LoRA/QLoRA 微调、离线评测报告;训练前增加蒸馏与 smoke 自检门禁。 \\
    学习状态追踪 & 学习事件流 + 课程画像/全局画像;将薄弱点、完成主题、学习时长等沉淀为可追踪档案。 \\
    \bottomrule
  \end{tabular}
\end{table}

接口层面,平台对外保持稳定的对话协议:以 \texttt{messages} 表示多轮对话,以 \texttt{mode} 表示技能选择;当模式名以 \texttt{\_rag} 结尾时,表示在保持技能风格不变的同时启用检索增强。对于需要工具调用的场景,采用结构化工具列表与 \texttt{tool} 回注消息组织对话,使“调用—执行—解释”链路可复现。

\section{数据与权限模型概述}
系统采用 JWT + RBAC 进行鉴权与授权:学生、教师、助教在课程、作业、学习画像等资源上具备不同权限边界。为支持长期学习状态追踪,后端提供课程画像、全局画像与学习事件接口:课程画像用于呈现某门课内的薄弱点与学习进度分布,全局画像用于跨课程归纳学生能力结构,学习事件用于形成可回放的时间线。对于知识库内容,系统在索引与检索环节引入课程维度的访问控制(ACL),避免跨课程材料泄露与越权引用。

\section{本章小结}
本章从应用场景出发给出系统需求与总体设计,明确平台需同时满足多模式辅导、可追溯检索、可验证执行、引导式学习、长期学习追踪以及训练评测闭环等能力,并给出分层架构与模块划分。下一章将进一步围绕 Skills、GraphRAG、工具调用、学习状态追踪与训练评测管线等关键模块展开实现细节。
