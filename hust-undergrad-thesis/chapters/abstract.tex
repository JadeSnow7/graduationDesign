\chapter*{摘要}
\addcontentsline{toc}{chapter}{摘要}
本文面向高等教育中"课程类型多样、学生差异显著、过程性反馈成本高、生成式模型可信度难以保障"等问题,设计并实现一套以学生为中心的智能教学平台。平台以学习过程数据为核心,围绕作业提交、对话辅导与学习事件形成可追踪的学生档案,并在此基础上提供可控、可追溯、可迭代的智能辅导能力。系统采用前后端分离与服务化架构:前端使用 React + TypeScript + Vite 构建 Web/H5 客户端,可嵌入企业微信 WebView;后端使用 Go + Gin 提供业务 API、JWT 鉴权与 RBAC 权限管理;AI 服务基于 Python + FastAPI 负责编排对话、写作分析、引导式学习、工具调用与检索增强,通过 OpenAI-compatible 接口对接推理服务。异构加速层支持 GPU、NPU 与 FPGA 三种后端:GPU/NPU 用于大模型推理,FPGA 用于 Embedding 服务的低延迟加速与资源解耦,在保持检索质量的前提下显著降低能耗。为降低幻觉并提升可复核性,平台引入 GraphRAG 检索增强,将课程资料构建为图结构索引,生成回答时注入证据片段并按编号引用;对于数值计算、仿真或格式检查等可验证任务,系统通过工具调用将关键结论锚定在可执行结果上。为支持模型定制与持续迭代,平台提供数据规范、LoRA/QLoRA 微调与离线评测脚本,并引入数据蒸馏与 smoke 验证用于训练链路自检。本文以电磁场推导型课程与研究生专业英文写作课程作为示例场景进行验证,展示平台在不同课程中的可迁移性与工程可落地性。

\vspace{0.5cm}
\noindent\textbf{关键词:}智能教学平台;学生档案;GraphRAG;工具调用;引导式学习;异构加速;FPGA
\clearpage
