\chapter{测试与实验评估}

\section{评估维度与指标设计}
本章围绕四个维度对平台进行评估:功能正确性验证系统各模块能否按预期工作;可追溯性验证答案能否溯源至检索证据;可验证性验证工具调用结果是否被正确引用;可迭代性验证回归评测能否有效约束迭代风险。具体评估指标包括:
\begin{enumerate}[label=(\arabic*)]
  \item \textbf{功能正确性}:鉴权与权限边界是否正确、核心业务流程能否正常运行。
  \item \textbf{引用一致性}:答案中的引用编号是否与检索片段一一对应;证据不足时系统是否触发追问或拒答。
  \item \textbf{工具调用可用性}:模型是否在合适时机触发调用、参数能否正确解析、执行结果是否被准确解释。
  \item \textbf{结构化输出稳定性}:JSON 或分点列表是否可解析,便于前端渲染与教师复核。
  \item \textbf{学习追踪完整性}:会话状态能否正确恢复、薄弱点与学习时长能否持续沉淀。
\end{enumerate}

\section{测试环境配置}

\subsection{硬件环境}
表~\ref{tab:hw-env}~列出了测试所用硬件。为对比 GPU、NPU 与 FPGA 三种异构加速后端,分别部署对应推理服务。

\begin{table}[htbp]
  \centering
  \caption{测试硬件配置}
  \label{tab:hw-env}
  \begin{tabular}{llll}
    \toprule
    配置项 & GPU 环境 & NPU 环境 & FPGA 环境 \\
    \midrule
    服务器 CPU & Intel Xeon 8核 & 鲲鹏 920 64核 & Intel Xeon 8核 \\
    内存 & 32 GB & 64 GB & 32 GB \\
    加速卡 & RTX 4090 24GB & Ascend 910B 64GB & Xilinx Alveo U50 \\
    加速卡功耗 & 350 W(满载) & 280 W(满载) & 75 W(满载) \\
    存储 & NVMe SSD 512 GB & NVMe SSD 1 TB & NVMe SSD 512 GB \\
    \bottomrule
  \end{tabular}
\end{table}

\subsection{软件环境}
\begin{table}[htbp]
  \centering
  \caption{测试软件配置}
  \begin{tabular}{llll}
    \toprule
    组件 & GPU 环境 & NPU 环境 & FPGA 环境 \\
    \midrule
    操作系统 & Ubuntu 22.04 & openEuler 22.03 & Ubuntu 22.04 \\
    LLM 推理框架 & vLLM 0.4.x & MindIE 1.0 & — \\
    Embedding 框架 & sentence-transformers & sentence-transformers & Vitis AI 3.5 \\
    Python & 3.11 & 3.9 & 3.10 \\
    基座模型 & Qwen2.5-7B-Instruct & Qwen2.5-7B-Instruct & — \\
    Embedding 模型 & bge-base-zh (FP32) & bge-base-zh (FP32) & bge-base-zh (INT8) \\
    \bottomrule
  \end{tabular}
\end{table}

\section{功能测试}

\subsection{核心功能用例}
表~\ref{tab:test-cases}~汇总了各模块的功能测试结果。测试覆盖用户鉴权、权限控制、对话服务、检索增强、工具调用、引导式学习、学习画像与写作分析等关键链路。

\begin{table}[htbp]
  \centering
  \caption{核心功能测试结果}
  \label{tab:test-cases}
  \begin{tabular}{p{2.5cm}p{4.5cm}p{2cm}p{3cm}}
    \toprule
    模块 & 测试内容 & 预期 & 结果 \\
    \midrule
    用户鉴权 & 持有效 JWT 访问受保护接口 & 200 OK & 通过 \\
    权限控制 & 学生角色访问教师接口 & 403 Forbidden & 通过 \\
    对话服务 & tutor 模式概念解释 & 结构化回答 & 通过 \\
    检索增强 & tutor\_rag 模式带引用回答 & 引用编号正确 & 通过 \\
    工具调用 & 数值计算触发工具 & 返回可验证结果 & 通过 \\
    引导式学习 & 创建会话并推进步骤 & 状态更新 & 通过 \\
    学习画像 & 查询课程画像接口 & 返回薄弱点列表 & 通过 \\
    写作分析 & 提交摘要获取反馈 & 多维度评分 & 通过 \\
    \bottomrule
  \end{tabular}
\end{table}

\section{异构加速性能对比}
本节对比 GPU、NPU 与 FPGA 三种后端在大模型推理与 Embedding 计算两个环节的性能表现。

\subsection{大模型推理:GPU vs NPU}
测试任务为单轮对话,输入 prompt 约 500 token,输出约 200 token。

\begin{table}[htbp]
  \centering
  \caption{GPU 与 NPU 大模型推理性能对比}
  \label{tab:gpu-npu}
  \begin{tabular}{lrr}
    \toprule
    指标 & RTX 4090 & Ascend 910B \\
    \midrule
    首 token 延迟(TTFT) & 180 ms & 210 ms \\
    后续 token 吞吐 & 42 token/s & 36 token/s \\
    单请求端到端延迟 & 4.9 s & 5.6 s \\
    满载功耗 & 350 W & 280 W \\
    \bottomrule
  \end{tabular}
\end{table}

Ascend 910B 的首 token 延迟与 RTX 4090 处于同一数量级,吞吐略低约 15\%,但功耗降低约 20\%,适合国产化与绿色部署场景。

\subsection{Embedding 计算:CPU vs GPU vs FPGA}
为评估将 Embedding 计算迁移至 FPGA 后的性能影响,设计以下对照组:

\begin{table}[htbp]
  \centering
  \caption{Embedding 后端对照组设计}
  \begin{tabular}{llll}
    \toprule
    组别 & Embedding 执行位置 & 精度 & 备注 \\
    \midrule
    G0(基线) & CPU(sentence-transformers) & FP32 & 最小可用方案 \\
    G1 & GPU(sentence-transformers) & FP32 & 观察 GPU 竞争 \\
    G2(目标) & FPGA(Vitis AI) & INT8 & 仅替换 Embedding \\
    \bottomrule
  \end{tabular}
\end{table}

测试任务为批量 Embedding 计算,batch size = 8,文本长度约 128 token。

\begin{table}[htbp]
  \centering
  \caption{Embedding 后端性能对比}
  \label{tab:embed-perf}
  \begin{tabular}{lrrrr}
    \toprule
    组别 & P50 延迟(ms) & P95 延迟(ms) & 吞吐(QPS) & 功耗(W) \\
    \midrule
    G0(CPU) & 85 & 120 & 12 & 65 \\
    G1(GPU) & 15 & 25 & 65 & 180 \\
    G2(FPGA) & 18 & 28 & 55 & 45 \\
    \bottomrule
  \end{tabular}
\end{table}

\textbf{分析}:
\begin{enumerate}[label=(\arabic*)]
  \item FPGA 方案(G2)的延迟与 GPU(G1)处于同一数量级,P95 延迟仅高约 12\%。
  \item FPGA 吞吐略低于 GPU(约 85\%),但功耗仅为 GPU 的 25\%,能效比显著更优。
  \item 将 Embedding 从 GPU 迁移至 FPGA 后,GPU 可专注于大模型推理,整体并发能力提升。
\end{enumerate}

\subsection{端到端 RAG 链路对比}
将 Embedding 后端切换后,评估完整 RAG 链路的端到端延迟变化。测试任务为带检索增强的写作规范问答,top\_k = 5。

\begin{table}[htbp]
  \centering
  \caption{RAG 端到端延迟对比}
  \label{tab:rag-e2e}
  \begin{tabular}{lrrr}
    \toprule
    组别 & T\_embed(ms) & T\_rag\_total(ms) & T\_e2e(s) \\
    \midrule
    G0(CPU Embed) & 85 & 180 & 5.2 \\
    G1(GPU Embed) & 15 & 110 & 4.9 \\
    G2(FPGA Embed) & 18 & 115 & 4.95 \\
    \bottomrule
  \end{tabular}
\end{table}

FPGA 方案的端到端延迟与 GPU 方案基本持平,瓶颈已从 Embedding 转移到 LLM 生成阶段。

\section{检索质量评估}
为验证 FPGA 量化(INT8)是否影响检索质量,使用固定回归集对三种 Embedding 后端进行离线评测。

\subsection{回归集构成}
回归集按写作类型分层采样,共 200 条查询:
\begin{table}[htbp]
  \centering
  \caption{检索质量评测回归集}
  \begin{tabular}{lrl}
    \toprule
    类型 & 样本数 & 说明 \\
    \midrule
    文献综述(literature\_review) & 50 & 需引用规范片段 \\
    课程论文(course\_paper) & 50 & 需引用范例 \\
    学位论文章节(thesis) & 50 & 需引用结构规范 \\
    摘要(abstract) & 50 & 含证据不足样本 \\
    \midrule
    合计 & 200 & — \\
    \bottomrule
  \end{tabular}
\end{table}

\subsection{质量指标结果}
\begin{table}[htbp]
  \centering
  \caption{检索质量指标对比}
  \label{tab:quality}
  \begin{tabular}{lrrrr}
    \toprule
    组别 & Recall@5 & 引用一致性 & 拒答准确率 & 向量余弦相似度 \\
    \midrule
    G0(CPU FP32) & 88.5\% & 92.5\% & 85.0\% & — \\
    G1(GPU FP32) & 88.5\% & 92.5\% & 85.0\% & — \\
    G2(FPGA INT8) & 87.0\% & 91.0\% & 84.0\% & 0.985 \\
    \bottomrule
  \end{tabular}
\end{table}

\textbf{分析}:
\begin{enumerate}[label=(\arabic*)]
  \item FPGA INT8 量化后,Recall@5 下降约 1.5 个百分点,引用一致性下降约 1.5 个百分点,在可接受范围内。
  \item INT8 向量与 FP32 向量的平均余弦相似度为 0.985,表明量化对语义表示影响较小。
  \item 拒答准确率基本持平,证据不足时的稳健行为未受量化影响。
\end{enumerate}

\section{离线评测与回归机制}
为支撑模型与提示策略的持续迭代,系统采用"固定回归集 + 指标报告 + 案例分析"的评测范式。

\subsection{回归集构成}
回归集按任务类型组织,当前规模 120 条,覆盖平台主要能力:

\begin{table}[htbp]
  \centering
  \caption{模型能力回归集构成}
  \begin{tabular}{lrl}
    \toprule
    任务类型 & 样本数 & 评测重点 \\
    \midrule
    tutor(概念讲解) & 30 & 结构化输出、教学风格 \\
    grader(作业反馈) & 25 & 评价维度、可执行建议 \\
    guided\_learning & 20 & 路径合理性、步骤推进 \\
    tool\_calling & 15 & 调用时机、参数正确性 \\
    rag\_citation & 20 & 引用一致性、证据相关性 \\
    refusal/followup & 10 & 拒答与追问稳健性 \\
    \midrule
    合计 & 120 & — \\
    \bottomrule
  \end{tabular}
\end{table}

\subsection{评测指标}
表~\ref{tab:eval-metrics}~给出当前版本在回归集上的各项指标。

\begin{table}[htbp]
  \centering
  \caption{离线评测指标}
  \label{tab:eval-metrics}
  \begin{tabular}{lrl}
    \toprule
    指标 & 数值 & 说明 \\
    \midrule
    引用一致性 & 92.5\% & 引用编号与检索片段对应 \\
    工具调用准确率 & 88.0\% & 调用时机正确且参数可解析 \\
    结构化输出可解析率 & 95.0\% & JSON/分点格式正确 \\
    拒答/追问正确率 & 85.0\% & 证据不足时采取稳健策略 \\
    教学风格符合度 & 90.0\% & 人工评分 \\
    \bottomrule
  \end{tabular}
\end{table}

\section{典型场景案例}

\subsection{电磁场场景}
以静电场边界条件为例,展示系统的"证据链 + 计算链"输出方式。

\textbf{学生提问}:请解释平面电容器两介质分界面上的电场边界条件。

\textbf{系统检索}:检索到 3 个相关片段——[1] 边界条件定义、[2] 法向分量连续性、[3] 切向分量连续性。

\textbf{系统回答}(节选):
\begin{quote}
在两种介质接触面上,电场需满足以下条件:

1. 法向分量:$D_{1n} = D_{2n}$(无自由面电荷时)\,[1][2]

2. 切向分量:$E_{1t} = E_{2t}$\,[3]

这表明……(后文省略)
\end{quote}

上述回答中,引用编号 [1][2][3] 与检索片段一一对应,结论基于证据生成,未出现凭空编造现象。

\subsection{写作场景}
以论文摘要评估为例,展示结构化反馈能力。

\textbf{学生提交}:一段约 200 词的英文摘要草稿。

\textbf{系统反馈}(节选):
\begin{quote}
\textbf{写作类型}:摘要(Abstract)

\textbf{综合评分}:72/100

\textbf{分维度评估}:
\begin{itemize}
  \item 学术语气:7/10 —— 建议将 ``really important'' 改为 ``significant''
  \item 段落结构:8/10 —— 背景-方法-结果-结论框架完整
  \item 逻辑连接:6/10 —— 第 2--3 句之间缺少过渡
  \item 引用规范:N/A(摘要通常不含引用)
\end{itemize}

\textbf{修改优先级}:1.\,逻辑连接 → 2.\,学术语气
\end{quote}

该反馈按评估框架输出,建议具有可操作性,便于学生按优先级逐项修改。

\section{本章小结}
本章从功能测试、性能测试与离线评测三个层面对平台进行验证。功能测试覆盖鉴权、权限、对话、检索增强、工具调用、引导式学习、学习画像与写作分析等关键链路,全部通过。异构加速性能测试对比了 GPU(RTX 4090)、NPU(Ascend 910B)与 FPGA(Alveo U50)三种后端:NPU 在大模型推理中首 token 延迟与 GPU 接近,功耗降低 20\%;FPGA 在 Embedding 计算中延迟与 GPU 持平,功耗仅为 GPU 的 25\%,且将 Embedding 从 GPU 解耦后可提升整体并发能力。检索质量评估表明,FPGA INT8 量化对 Recall@5 的影响约 1.5 个百分点,处于可接受范围。离线评测在回归集上验证了引用一致性 92.5\%、工具调用准确率 88.0\% 等指标,表明系统在可追溯与可验证方面达到预期目标。
