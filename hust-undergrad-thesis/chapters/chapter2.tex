\chapter{相关技术与理论基础}

\section{前端技术}
前端采用 Vue 3 + Vite + TypeScript 技术栈,以组件化方式构建企业微信内嵌 H5 应用。Vue 3 的 Composition API 提供更灵活的逻辑组织方式,Vite 基于 ESM 的热更新机制显著提升开发效率。应用需适配企业微信 WebView 与普通浏览器的差异,采用响应式布局与特性检测确保跨终端兼容性。

\section{后端技术}
后端采用 Go + Gin 框架构建 RESTful API 服务,采用分层架构(Handler/Service/Repository)实现关注点分离。权限控制采用 RBAC 模型,结合 JWT 实现无状态认证,支持教师、助教、学生等多角色细粒度权限管理。数据库采用 MySQL,通过迁移工具管理表结构变更,确保开发与部署环境一致性。

\section{大语言模型与检索增强}

\subsection{Transformer 与 LLM 基础}
Transformer 架构\cite{vaswani2017}通过自注意力机制实现序列建模,是当前大语言模型的基础。预训练语言模型(如 GPT、Qwen)通过大规模文本语料学习语言知识,展现出强大的自然语言理解与生成能力,但存在幻觉、推理可信度等问题。

\subsection{传统 RAG 原理与局限}
检索增强生成(RAG)\cite{lewis2020}通过向量检索将外部知识注入模型上下文,有效降低幻觉并提升回答依据。传统 RAG 流程包括文档切片、向量化、相似度检索与上下文拼接。然而,纯向量检索难以捕获知识间的结构化关联,对于理工科推导型问题中的公式推导链、概念依赖关系表达能力不足。

\subsection{GraphRAG 核心思想}
GraphRAG 在传统 RAG 基础上引入知识图谱结构,将文档组织为"节点-边-片段"的图结构索引,支持多跳关联检索与图扩展推理。其核心机制包括:
\begin{enumerate}[label=(\arabic*)]
  \item \textbf{混合检索}:结合关键词匹配(Bigram)与语义向量检索,通过 RRF(Reciprocal Rank Fusion)融合排序,兼顾精确匹配与语义相关。
  \item \textbf{图扩展推理}:从检索到的种子片段出发,沿知识图谱边扩展 1-2 跳,获取前置概念与相关公式,构建完整推导上下文。
  \item \textbf{来源追溯}:回答中使用 \texttt{[1][2]} 标注引用来源,防止幻觉并提升可信度。
\end{enumerate}

\section{工具调用与函数执行}
工具调用(Tool Calling / Function Calling)使大语言模型能够识别何时需要外部工具,并生成结构化的调用请求。该机制对于理工科问答具有重要价值:
\begin{enumerate}[label=(\arabic*)]
  \item \textbf{符号计算}:调用 SymPy 等符号计算库进行公式推导与验证。
  \item \textbf{数值仿真}:调用 Python 数值计算服务执行电磁场仿真。
  \item \textbf{表达式求值}:对物理常数与参数进行精确计算,避免模型"心算"错误。
\end{enumerate}
工具调用流程包括:模型判断是否需要工具 $\rightarrow$ 生成调用请求 $\rightarrow$ 执行工具获取结果 $\rightarrow$ 将结果注入上下文 $\rightarrow$ 模型基于结果生成最终回答。该机制确保数值计算的可验证性与可复现性。

\section{后训练与领域适配}
后训练(Post-Training)通过在预训练模型基础上进行领域数据微调,提升模型在特定任务上的表现。常用方法包括:
\begin{enumerate}[label=(\arabic*)]
  \item \textbf{继续预训练(Continue Pre-training)}:使用领域语料进一步训练,强化专业术语与公式符号的理解。
  \item \textbf{监督微调(SFT)}:使用高质量问答对进行指令微调,适配教学风格与回答格式。
  \item \textbf{LoRA 微调}:低秩适配方法,仅训练约 0.1\% 参数,单 GPU 可完成训练,适合资源受限场景。
  \item \textbf{工具调用 SFT}:使用包含工具调用标注的数据集训练模型正确识别调用时机。
\end{enumerate}
评估指标包括教学质量(人工评分)、工具调用准确率、引用正确率、计算正确率与幻觉率。

\section{数值仿真与可视化}
电磁场数值仿真采用有限差分法(FDM)求解典型模型,如二维静电场(Laplace/Poisson 方程)、同轴线电容场分布等。Python 科学计算生态(NumPy/SciPy/Matplotlib)提供高效的数值计算与可视化能力。仿真服务通过 FastAPI 暴露 HTTP 接口,支持参数化任务提交与异步计算,输出包括字段数据(JSON)与可视化图(PNG)。仿真结果可作为可验证证据注入 AI 回答,形成"计算-解释-理解"的教学闭环。

\section{本章小结}
本章介绍了系统涉及的关键技术与理论基础,包括前端 Vue 3 技术栈、后端 Go/Gin 框架与 RBAC 权限模型、大语言模型与 GraphRAG 检索增强、工具调用机制、后训练方法以及数值仿真技术。这些技术共同支撑"工具调用 + GraphRAG + 后训练"三位一体辅助推理框架的实现。
