\chapter{总结与展望}
本文围绕《电磁场与电磁波》课程知识结构复杂与答疑负担重等问题,完成了课程知识图谱构建、大语言模型与知识图谱协同问答引擎设计以及系统原型实现。原型采用企业微信内嵌 H5/网页作为入口,后端提供统一鉴权与权限控制,问答服务通过检索增强与可控提示模板提高回答的可追溯性与安全性,并形成了可部署的系统架构与接口方案。

后续工作可从以下方面继续推进:一是完善知识抽取与融合策略,引入实体关系抽取与图数据库,提高知识图谱的细粒度与可维护性;二是构建更系统的课程问答评测集,量化准确率、引用一致性与教学效果;三是优化检索与上下文压缩策略,降低响应时延;四是完善企业微信 OAuth 与消息推送能力,提升真实教学场景下的使用体验。
