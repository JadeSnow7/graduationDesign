\chapter{总结与展望}

\section{工作回顾}
本文针对多课程教学场景中"反馈成本较高、学习过程难以追踪、生成式模型可信度不足"等问题,设计并实现了一套以学生为中心的智能教学平台。下面从四个方面对所做工作进行回顾。

\subsection{系统架构与工程实现}
平台采用前后端分离与服务化架构:前端以 React + TypeScript + Vite 构建 Web/H5 客户端,可嵌入企业微信 WebView 并保持跨端一致体验;后端基于 Go + Gin 提供业务 API,使用 JWT 进行无状态认证,配合 RBAC 模型实现多角色权限管理;AI 服务以 Python + FastAPI 实现,通过 OpenAI-compatible 接口与上游推理服务对接。代码采用 Monorepo 组织,前后端共享类型定义,降低接口不一致风险。在异构加速方面,系统支持 GPU、NPU 与 FPGA 三种后端——GPU/NPU 用于大模型推理,FPGA 用于 Embedding 服务的低延迟加速与资源解耦。

\subsection{可追溯与可验证的智能辅导}
为应对大语言模型的幻觉与可信度问题,平台构建了"证据链 + 计算链"的双重可追溯机制:
\begin{enumerate}[label=(\arabic*)]
  \item \textbf{GraphRAG 检索增强}:将课程讲义与规范构建为图结构索引,采用混合检索与图扩展获取相关证据,回答中以编号形式标注来源,使结论可追溯。
  \item \textbf{工具调用}:对数值计算、仿真与格式检查等任务,通过工具调用获取可执行结果,降低模型"心算失误"与"凭空建议"的风险。
\end{enumerate}

\subsection{过程性辅导与学习追踪}
平台以引导式学习为核心交互方式,将复杂学习主题拆解为可管理的步骤,逐步提问推进,实现"诊断—引导—巩固"的过程性辅导。同时,通过学习事件流与多级学习画像(课程画像/全局画像)将薄弱点、学习时长与完成主题等信息沉淀为可查阅档案,为教师侧学情分析与个性化干预提供数据支撑。

\subsection{训练与评测链路}
为支持模型定制与持续迭代,平台提供完整的后训练工具链:数据规范化确保训练与上线协议一致;数据蒸馏与 smoke 验证作为质量门禁;LoRA/QLoRA 参数高效微调输出可版本化的 adapter;离线回归评测通过固定评测集量化引用一致性、工具调用准确率与结构化输出稳定性等指标。

\section{主要贡献}
本文的主要贡献可归纳为以下五点:
\begin{enumerate}[label=(\arabic*)]
  \item 提出并落地"学生中心数据闭环 + GraphRAG 可追溯 + 工具调用可验证 + 引导式学习"的一体化框架,形成可复用的工程路径。
  \item 设计"通用能力 + 课程专属模块"的模块化策略,以电磁场与研究生专业英文写作为例验证跨课程可迁移性。
  \item 对比 GPU 与 NPU 两种大模型推理后端的性能与能耗,为校园场景下的国产化与绿色部署提供参考。
  \item 引入 FPGA 加速 Embedding 服务,实现检索链路与大模型推理的资源解耦,在保持检索质量的前提下显著降低能耗。
  \item 建立面向持续迭代的训练与回归评测流程,使模型迭代从主观判断转向可量化回归。
\end{enumerate}

\section{不足与改进方向}
尽管本文完成了平台原型实现,仍存在以下局限:

\subsection{当前不足}
\begin{enumerate}[label=(\arabic*)]
  \item \textbf{知识图谱粒度}:当前 GraphRAG 以文档片段为节点,对概念、公式与物理量之间细粒度关系的建模仍有欠缺。
  \item \textbf{评测集规模}:回归集约百条量级,难以覆盖所有边界情况;需进一步扩展并引入人工评测。
  \item \textbf{真实场景验证}:当前验证主要基于模拟数据,尚未在真实课程中大规模部署并收集师生反馈。
  \item \textbf{企业微信深度集成}:OAuth 与消息推送能力处于预留状态,未完成完整的内嵌体验。
  \item \textbf{FPGA 量化损失}:INT8 量化导致 Recall@5 下降约 1.5 个百分点,对高精度场景需进一步优化量化策略。
\end{enumerate}

\subsection{后续工作}
针对上述不足,可从以下方向继续改进:
\begin{enumerate}[label=(\arabic*)]
  \item \textbf{知识抽取与融合}:引入实体关系抽取与图数据库(如 Neo4j),提升知识图谱的细粒度与可维护性。
  \item \textbf{评测体系扩展}:构建千条量级的课程问答评测集,量化准确率与教学效果;引入 A/B 测试框架。
  \item \textbf{检索与上下文优化}:优化检索排序与上下文压缩策略,降低延迟并提升长上下文任务的稳定性。
  \item \textbf{企业微信完整集成}:完成 OAuth 认证与消息推送能力,提升触达体验。
  \item \textbf{多模态能力}:引入图像理解,支持手写作业识别与公式图片解析。
  \item \textbf{学习分析可视化}:为教师提供班级学情看板,展示薄弱点分布与能力发展趋势。
  \item \textbf{异构加速生态完善}:跟进 Ascend 与 Xilinx 软件栈更新,探索 Reranker 的 FPGA 加速与混合精度量化。
\end{enumerate}

\section{结语}
本文完成了一套以学生为中心的智能教学平台原型,在"可追溯、可验证、可迭代"的设计目标下,为将大语言模型可控地引入教育场景提供了工程实践参考。通过引入 GPU、NPU 与 FPGA 三种异构加速后端,平台在性能、能耗与资源利用之间取得了更好的平衡。希望本研究能够为高校智能教学系统的建设提供借鉴,并在后续工作中不断完善与推广。
