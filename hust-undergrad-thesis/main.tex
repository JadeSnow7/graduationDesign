% !TEX program = xelatex
\documentclass[UTF8,zihao=-4,oneside,fontset=fandol]{ctexbook}

\usepackage[a4paper,top=2.5cm,bottom=2.5cm,left=3cm,right=2.5cm]{geometry}
\usepackage{setspace}
\usepackage{graphicx}
\usepackage{booktabs}
\usepackage{tabularx}
\usepackage{array}
\usepackage{amsmath,amssymb}
\usepackage{hyperref}
\usepackage{fancyhdr}
\usepackage{caption}
\usepackage{subcaption}
\usepackage{enumitem}
\usepackage{csquotes}
\usepackage[
  backend=biber,
  style=gb7714-2015,
  sorting=none
]{biblatex}
\addbibresource{references.bib}

\setlength{\parindent}{2em}
\setlength{\parskip}{0pt}
\onehalfspacing

\pagestyle{fancy}
\fancyhf{}
\fancyhead[LE,RO]{\thepage}
\fancyhead[LO]{\nouppercase{\leftmark}}
\fancyhead[RE]{\nouppercase{\rightmark}}
\renewcommand{\headrulewidth}{0.4pt}
\setlength{\headheight}{14.5pt}

\newcommand{\HUSTTitle}{基于大模型的以学生为中心的智能教学平台设计与实现}
\newcommand{\HUSTSubtitle}{——以本科生《电磁场与电磁波》和研究生《学术规范与论文写作》为例}
\newcommand{\HUSTAuthor}{胡傲东}
\newcommand{\HUSTStudentId}{U202214900}
\newcommand{\HUSTMajor}{基础电路设计与集成系统}
\newcommand{\HUSTSchool}{集成电路学院}
\newcommand{\HUSTAdvisor}{聂彦}
\newcommand{\HUSTDate}{2026年5月}

\hypersetup{
  unicode=true,
  pdftitle={\HUSTTitle\HUSTSubtitle},
  pdfauthor={\HUSTAuthor},
  pdfsubject={本科毕业论文},
  pdfkeywords={大语言模型,智能教学平台,学生中心,课程模块化,GraphRAG,检索增强生成,引导式学习,工具调用,LoRA,QLoRA,数据蒸馏,React,Go,FastAPI},
  hidelinks
}

\newcommand{\makecover}{%
  \thispagestyle{empty}%
  \begin{center}
    \vspace*{1.5cm}
    {\zihao{2}\bfseries 华中科技大学本科毕业论文\par}
    \vspace{1cm}
    {\zihao{-2}\bfseries \HUSTTitle\par}
    \ifx\HUSTSubtitle\empty\else
      \vspace{0.5cm}
      {\zihao{3}\bfseries \HUSTSubtitle\par}
    \fi
  \end{center}
  \vfill
  \begin{center}
    \renewcommand{\arraystretch}{1.5}
    \begin{tabular}{rl}
      学生姓名: & \HUSTAuthor \\
      学\hspace{2em}号: & \HUSTStudentId \\
      专\hspace{2em}业: & \HUSTMajor \\
      学\hspace{2em}院: & \HUSTSchool \\
      指导教师: & \HUSTAdvisor \\
    \end{tabular}
  \end{center}
  \vfill
  \begin{center}
    \HUSTDate
  \end{center}
  \clearpage
}

\begin{document}
\frontmatter
\makecover
\chapter*{原创性声明}
\addcontentsline{toc}{chapter}{原创性声明}
本人声明所呈交的毕业论文是本人在导师指导下进行的研究成果。除文中已经注明引用的内容外,本论文不包含任何他人已经发表或撰写的研究成果。对本文的研究做出重要贡献的个人和集体,均已在文中以明确方式注明。

\vspace{2cm}
\begin{flushright}
作者签名:\underline{\hspace{4cm}}\\[0.5cm]
日期:\underline{\hspace{4cm}}
\end{flushright}

\clearpage
\chapter*{授权使用声明}
\addcontentsline{toc}{chapter}{授权使用声明}
本人同意华中科技大学保存并向国家有关部门或机构送交本论文的复印件和电子版,允许本论文被查阅和借阅。本人授权华中科技大学将本论文的全部或部分内容编入有关数据库进行检索,可以采用影印、缩印或其他复制手段保存和汇编本论文。

\vspace{2cm}
\begin{flushright}
作者签名:\underline{\hspace{4cm}}\\[0.5cm]
日期:\underline{\hspace{4cm}}
\end{flushright}
\clearpage

\chapter*{摘要}
\addcontentsline{toc}{chapter}{摘要}
本文面向高等教育中"课程类型多样、学生差异显著、过程性反馈成本高、生成式模型可信度难以保障"等问题,设计并实现一套以学生为中心的智能教学平台。平台以学习过程数据为核心,围绕作业提交、对话辅导与学习事件形成可追踪的学生档案,并在此基础上提供可控、可追溯、可迭代的智能辅导能力。系统采用前后端分离与服务化架构:前端使用 React + TypeScript + Vite 构建 Web/H5 客户端,可嵌入企业微信 WebView;后端使用 Go + Gin 提供业务 API、JWT 鉴权与 RBAC 权限管理;AI 服务基于 Python + FastAPI 负责编排对话、写作分析、引导式学习、工具调用与检索增强,通过 OpenAI-compatible 接口对接推理服务。异构加速层支持 GPU、NPU 与 FPGA 三种后端:GPU/NPU 用于大模型推理,FPGA 用于 Embedding 服务的低延迟加速与资源解耦,在保持检索质量的前提下显著降低能耗。为降低幻觉并提升可复核性,平台引入 GraphRAG 检索增强,将课程资料构建为图结构索引,生成回答时注入证据片段并按编号引用;对于数值计算、仿真或格式检查等可验证任务,系统通过工具调用将关键结论锚定在可执行结果上。为支持模型定制与持续迭代,平台提供数据规范、LoRA/QLoRA 微调与离线评测脚本,并引入数据蒸馏与 smoke 验证用于训练链路自检。本文以电磁场推导型课程与研究生专业英文写作课程作为示例场景进行验证,展示平台在不同课程中的可迁移性与工程可落地性。

\vspace{0.5cm}
\noindent\textbf{关键词:}智能教学平台;学生档案;GraphRAG;工具调用;引导式学习;异构加速;FPGA
\clearpage

\chapter*{Abstract}
\addcontentsline{toc}{chapter}{Abstract}
This thesis addresses common challenges in higher education, including diverse course types, significant learner differences, the high cost of process-oriented feedback, and the difficulty of guaranteeing reliability in generative models. We design and implement a student-centric intelligent teaching platform powered by large language models (LLMs). The platform organizes learning-process data (submissions, tutoring dialogues, and learning events) into longitudinal student profiles, enabling controllable, traceable, and iterative tutoring. The system follows a service-oriented architecture with a separated front end and back end: the client is built with React, TypeScript, and Vite for web/H5 (embeddable in a WeCom WebView); the backend uses Go and Gin to provide business APIs with JWT-based authentication and RBAC; and the AI service is implemented with Python and FastAPI to orchestrate chat, writing analysis, guided learning, tool calling, and retrieval augmentation, connecting to upstream inference via an OpenAI-compatible API. The heterogeneous acceleration layer supports GPU, NPU (Huawei Ascend), and FPGA (Xilinx Alveo) backends: GPU/NPU for LLM inference, and FPGA for low-latency embedding acceleration and resource decoupling, significantly reducing power consumption while maintaining retrieval quality. To reduce hallucinations and improve auditability, we introduce GraphRAG: course materials are indexed as a graph-structured knowledge base, and responses are grounded on retrieved evidence with explicit citations. For verifiable tasks such as numerical calculation/simulation or format checks, tool calling delegates key steps to executable tools and injects results back into the dialogue. To support model customization and continuous iteration, we provide a data specification, LoRA/QLoRA fine-tuning and offline evaluation scripts, as well as data distillation and smoke validation for pipeline sanity checks. Two representative scenarios---an electromagnetics derivation-intensive course and a graduate academic writing course---are used to demonstrate portability and engineering feasibility.

\vspace{0.5cm}
\noindent\textbf{Keywords:} intelligent teaching platform; student profile; GraphRAG; tool calling; guided learning; heterogeneous acceleration; FPGA
\clearpage

\tableofcontents
\listoffigures
\listoftables

\mainmatter
\chapter{绪论}
\section{研究背景与意义}
《电磁场与电磁波》课程是电子信息类、通信工程等专业的核心基础课程,知识点抽象、公式推导多且逻辑链条长,学生在自学与复习过程中容易出现概念混淆与步骤缺失。传统课堂与常规教学平台在即时答疑、过程性引导与知识结构呈现方面能力有限,教师在作业批改、答疑与学情分析中投入大量重复性工作。近年来,Transformer 架构\cite{vaswani2017}推动大语言模型在自然语言理解与生成方面取得突破,对话式模型(如 ChatGPT)\cite{openai2022}显示出辅助教学与答疑的潜力,但其幻觉与可解释性问题仍需外部知识约束。检索增强生成(RAG)\cite{lewis2020}与知识图谱\cite{hogan2020}提供了结构化知识支撑与可追溯机制,为构建面向课程教学的智能问答系统提供了新的技术路径。基于企业微信的 H5 入口具备便捷触达与组织管理优势,适合承载课程智能教学平台原型。

\section{国内外研究现状}
在教育信息化领域,学习管理系统与学习分析工具已实现课程资源管理、作业统计与学习过程记录,但对推导型课程的概念解释与解题引导支持不足。大语言模型在教育问答、自动反馈等场景中应用快速增长,研究重点从“能回答”转向“可追溯、可验证”,RAG 被广泛用于降低幻觉并提升答案依据\cite{lewis2020}。知识图谱作为结构化语义组织方式,在教材知识组织、概念关联与搜索推荐中具有优势\cite{hogan2020},但与大模型深度协同的课程问答系统仍处于探索阶段。综上,面向《电磁场与电磁波》课程构建知识图谱并与大模型协同的问答引擎,具有明确的研究与应用价值。

\section{研究内容与任务要求}
本文围绕课程知识图谱与协同问答引擎的构建开展研究,主要内容与任务要求如下。

\subsection{课题内容}
\begin{enumerate}[label=(\arabic*)]
  \item 《电磁场与电磁波》课程知识图谱构建。
  \item 大语言模型与知识图谱的协同问答引擎设计。
  \item 系统原型实现与企业微信集成或网页实现。
  \item 系统测试与评估。
\end{enumerate}

\subsection{课题任务要求}
\begin{enumerate}[label=(\arabic*)]
  \item 深入理解大语言模型的基本原理及其应用范式,掌握至少一种主流 LLM 的 API 调用方法。
  \item 掌握知识图谱的构建流程,能够针对《电磁场与电磁波》内容进行知识抽取、融合与存储。
  \item 完成一个包含知识图谱管理、智能问答交互集成功能的完整系统原型。
  \item 系统应能准确回答课程相关的概念、公式、定理等基础问题,并能进行简单的习题求解引导。
  \item 完成毕业设计论文的撰写,论文应结构清晰、论证充分、代码和数据详实。
\end{enumerate}

\section{论文结构}
本文共分三章:第一章介绍研究背景、研究现状以及课题内容与任务要求;第二章给出需求分析、系统架构与关键模块设计实现;第三章总结工作并展望后续改进方向。

\chapter{方法与实现}
\section{需求分析与总体设计}
平台面向管理员、教师/助教与学生三类主要角色,核心需求可归纳为三条主线:其一是\textbf{过程性数据沉淀},能够围绕写作提交、修改与对话辅导记录学习事件,并形成长期可追踪的学生画像;其二是\textbf{写作类型感知的反馈},针对文献综述、课程论文、学位论文与摘要等不同类型提供差异化 rubric 与结构化建议;其三是\textbf{可控与可追溯},在大模型生成建议时提供证据引用与复核入口,降低幻觉带来的教学风险。

非功能需求方面,系统需要具备可扩展与可维护的工程结构(便于迭代模型与课程模块)、清晰的权限边界(避免越权访问与答案泄露)、以及跨终端一致的调用契约(避免 Web/Mobile 接口分叉)。基于上述需求,本文采用前后端分离与服务化架构,将教学业务、AI 能力与检索索引解耦,通过统一鉴权与模块门控策略保证能力可治理。

\section{系统架构}
系统整体采用“客户端—后端业务—AI 服务—检索/存储”的分层结构:客户端提供 Web 与移动端入口;后端提供业务 API、JWT 鉴权、RBAC 权限与课程模块门控;AI 服务负责提示模板、写作分析与对话能力的编排,并通过 OpenAI-compatible 接口调用上游大模型推理服务;GraphRAG 作为可选组件提供课程知识库检索与引用溯源。整体架构如图~\ref{fig:architecture} 所示。

\begin{figure}[htbp]
  \centering
  \fbox{\rule{0pt}{5cm}\rule{10cm}{0pt}}
  \caption{平台总体架构示意(占位)}
  \label{fig:architecture}
\end{figure}

\section{统一契约与跨端共享}
为避免跨端开发中出现“同一业务多套接口/字段”的问题,系统采用 Monorepo 组织代码,并抽取共享包沉淀 types 与统一 SDK。共享 SDK 负责:
(1) 统一请求层(鉴权头、错误归一、超时与重试策略);
(2) 以类型定义约束前后端契约,减少字段不一致导致的运行时错误;
(3) 为 Web/Mobile 提供一致的 API 调用方式,使平台差异主要集中在 UI 与交互层。
该设计降低了多端协作成本,并为后续在不同课程模块间复用能力提供基础。

\section{学生中心数据模型与学习事件}
平台以学生为中心组织数据。在课程层面,写作提交被建模为可追踪的业务对象:包含写作类型、标题与内容、提交时间、AI 分析结果与教师反馈等字段;在过程层面,系统记录关键学习事件(如写作提交、写作分析完成、对话辅导、学习时长心跳等),用于后续聚合形成学生画像。画像不仅包含“分数”,更强调可解释的能力维度(例如结构清晰度、证据使用、学术语气与引用规范),从而支持纵向对比与个性化干预。

\section{引导式学习与薄弱点追踪}
除“写作提交—分析—反馈”流程外,平台提供面向过程性辅导的引导式学习能力(guided learning):系统首先为某一学习主题生成 3--6 步的学习路径(learning path),随后以苏格拉底式提问引导学生逐步完成每一步。例如,在写作课程中,学习主题可围绕 thesis statement、段落结构、证据使用与引用规范等展开,系统会在每轮对话中只提出一个关键问题,并根据学生回答的完整性决定是否进入下一步,从而把复杂能力训练拆解为可管理的阶段。

实现上,AI 服务提供 \texttt{/v1/chat/guided} 端点,使用会话状态(session\_id)维护学习目标、当前步骤与路径结构,并在首轮由模型输出 JSON 路径以便前端渲染进度。为将对话信号沉淀为画像特征,系统在每次对话后对助教回复中的纠错与提示语句做轻量检测,提取与写作相关的薄弱点概念(如“逻辑连接”“引用规范”“论点展开”),记录到会话中并可同步到后端学习档案。结合 GraphRAG 时,系统会把检索到的课程规范与示例片段作为证据注入对话上下文,要求回答标注引用编号并在证据不足时追问或拒答,从而提升引导式建议的可追溯性与可复核性。

\section{写作类型感知的智能分析服务}
写作分析服务以“写作类型 + rubric + 结构化输出”为核心。服务端首先识别或校验写作类型,并选择对应的评估维度与权重;随后调用上游大模型生成反馈,并将输出解析为维度评分、优点与改进建议等结构化字段,便于前端展示与教师复核。与通用润色工具不同,本文更关注“可执行建议”:例如指出段落功能缺失、论证链条不完整、证据不足或引用格式问题,并给出可操作的修改方案。该设计使反馈更贴近课程要求,也更便于后续沉淀高质量标注数据。

\section{GraphRAG 知识库与检索增强生成}
为降低大模型在写作辅导中的幻觉风险,系统引入 GraphRAG 检索增强生成流程\cite{lewis2020}。针对研究生《学术规范与论文写作》课程中常见的“引用格式错误”与“学术不端风险”,本文构建了基于向量检索的课程知识库。构建过程如下:首先,将课程讲义、学校学位论文写作规范及优秀的历年范文进行结构化清洗,并按照“章节—段落”的层级进行切分(Chunking),默认切片大小设定为 1200 字符。其次,采用 `text-embedding-v3` 模型将切分后的文本片段转化为高维度向量(Embedding),并存储于 FAISS 向量数据库中。
当学生在对话中咨询关于引用规范或格式要求的问题时,系统首先将用户查询转化为向量,通过余弦相似度在向量数据库中检索最相关的 $K$ 个规范条款或范文片段。检索到的片段被作为“证据(Evidence)”注入到大模型的上下文提示词(Prompt)中,并强制要求模型仅依据检索到的证据回答,并在回答末尾标注引用来源编号。这一“向量化—检索—注入—生成”的闭环不仅显著降低了模型的幻觉风险,更模拟了真实的学术问题解决过程——即“查阅规范—理解条款—应用执行”,从而在技术实现的底层逻辑上契合了课程的教学目标。

\subsection{学术规范向量知识库构建过程}
针对《学术规范与论文写作》课程中“建议容易泛化、缺少课程依据”的问题,系统将课程规范资料与论文写作指南按统一流程构建为向量知识库。首先,离线 ingestion 支持 `.md/.markdown/.pdf/.txt` 多源文本输入;随后按章节与段落进行切分,并以 `--chunk-chars=1200` 作为默认分块上限,将原始文本转换为可检索的知识片段。接着,系统对片段执行 Embedding 向量化(`api|local|hash|env`,默认模型 `text-embedding-v3`),并将向量写入 `FAISS` 向量存储,同时维护图索引中的节点与邻接关系以支持图扩展检索。

在线推理阶段,查询先经过语义检索与关键词/图扩展召回,再将命中的证据片段注入提示词上下文,约束模型在证据范围内生成并标注引用编号。该“原文资料 $\rightarrow$ chunks $\rightarrow$ embeddings $\rightarrow$ vector store $\rightarrow$ 检索注入生成”的链路,将回答从“语言模型先验”转为“课程证据驱动”,可显著降低《学术规范与论文写作》辅导中的幻觉建议与领域知识缺失问题,并提升教师复核与过程追溯的可行性。

\section{工具调用与可验证能力}
除写作建议外,教学场景中仍存在需要“可验证计算/查询”的任务,例如对字数、结构要素或格式规则进行检查,或在理工类课程中进行数值计算与仿真。为此,AI 服务提供工具调用接口,使模型在需要精确结果时可调用外部工具并将结果回注到对话中,再生成解释性回答。工具调用能力本质上为系统提供了“外部可验证执行器”,用于约束模型的自由生成范围,降低“凭空计算/编造规则”的风险。本文在原型中实现了基础工具集合,并预留面向写作场景的扩展空间(如引用格式校验、结构要素检查等)。

\section{模型后训练与评测管线}
为使模型更贴近课程风格与任务需求,本文实现了面向写作/对话数据的后训练管线:包括数据规范、数据准备脚本、LoRA/QLoRA 微调脚本\cite{hu2021lora,dettmers2023qlora}与离线评测脚本。训练数据以多轮对话 JSONL 表示,并区分 tool/rag/style 等样本类型;评测阶段以固定回归集输出指标与案例,辅助迭代数据与提示策略。受数据规模与时间限制,本文先使用小规模样例数据完成端到端验证:训练脚本可稳定产出 adapter,评测脚本可输出困惑度、格式一致性与拒答准确率等指标,为后续在 Qwen3 8B 上进行 100k 规模训练提供工程基础。

为降低“数据格式不一致导致训练失败”的工程风险,本文在训练前增加了数据蒸馏与冒烟验证步骤:将 chat-style 的训练/评测 JSONL 通过 \texttt{scripts/ai/distill\_data.py} 蒸馏为 prompt/response 格式,并用 \texttt{scripts/ai/train\_smoke.py} 在分钟级输出困惑度等轻量指标,用于验证数据链路与指标输出链路可复现。需要强调的是,smoke 指标仅用于证明训练与评测链路可用,并不代表最终模型效果。

\begin{table}[htbp]
  \centering
  \caption{样例训练链路验证结果(用于证明训练与评测链路可用)}
  \label{tab:smoke-train}
  \begin{tabular}{lccp{6.2cm}}
    \toprule
    指标 & 训练集 & 验证集 & 说明 \\
    \midrule
    样本数 & 3 & 2 & 小规模 JSONL 样例数据,仅用于链路验证 \\
    Token 数 & 68 & 50 & 以分词后 token 计 \\
    困惑度(PPL) & 32.95 & 41.30 & 使用轻量模型完成端到端训练与评测,数值不代表最终效果 \\
    \bottomrule
  \end{tabular}
\end{table}

\subsection{阶段性训练结果同步(2026-02-08)}
在完成训练脚本与评测脚本的端到端连通验证后,项目于 2026-02-08 执行了首次 \texttt{all} 多任务训练评测(小样本)与随机三组回归测试,并将结果同步为统一事实源。指标如表~\ref{tab:stage-train-20260208} 所示。

\begin{table}[htbp]
  \centering
  \caption{阶段性训练结果(2026-02-08,同步批次)}
  \label{tab:stage-train-20260208}
  \begin{tabular}{lp{2.0cm}cccc}
    \toprule
    评测批次 & 样本规模 & key\_point\_coverage & refusal\_accuracy & response\_format & tool\_call\_accuracy \\
    \midrule
    首次 all 训练 & $n=5$ & 0.9167 & 0.8000 & 1.0000 & 0.0000 \\
    随机三组回归均值 & $3 \times n=6$ & 0.7333 & 0.7778 & 0.8333 & 0.0000 \\
    \bottomrule
  \end{tabular}
\end{table}

需要说明的是,上述结果仅用于证明“训练—评测—文档同步”链路可复现,属于阶段性验证数据,不作为本文正式实验结论。后续正式实验将在真实 \texttt{style/tool/rag} 数据闭环后重新训练并报告主结果。

\section{系统原型实现与企业微信集成}
平台前端采用 React + TypeScript 实现 Web 客户端,并提供基于 Expo 的移动端实现;后端采用 Go/Gin 提供课程、写作与学习事件相关 API,并通过 JWT 与 RBAC 实现权限治理;AI 服务基于 FastAPI,实现对话、写作分析、GraphRAG 与工具调用等能力,并对接 OpenAI-compatible 上游推理服务。原型部署层面提供 Docker Compose 配置以便快速启动与验证;对于企业微信等场景,系统预留 OAuth 与组织对接能力,以支持后续在真实教学流程中落地。

\section{系统测试与评估}
系统测试与评估围绕功能正确性、可追溯性与工程稳定性展开。功能测试验证写作提交与分析流程、权限校验、学习事件记录与查询等关键链路;可追溯性测试关注 RAG 模式下的引用一致性与“证据不足时追问/拒答”行为;工程测试关注典型请求的响应时间与服务稳定性。对于模型效果评估,本文采用“固定回归集 + 案例分析”的方式进行离线对比,并预留进一步的用户试用与课堂验证方案,用于在真实课程中评估建议的可采纳性与对学习效果的影响。

\chapter{系统需求分析与总体设计}

\section{应用场景与用户角色}
本项目面向高校多课程教学场景,强调以学生为中心的过程性支持与可追溯治理:既要覆盖理工科推导/计算密集任务(以电磁场为例),也要覆盖写作类“过程性训练 + 评价维度复杂”的任务(以研究生专业英文写作为例)。系统用户角色主要包括:学生(学习与提交)、教师/助教(发布与反馈、学情分析)以及管理员(课程与权限配置)。在上述角色协作下,平台需同时满足“日常答疑/辅导”的即时交互需求与“长期学习状态沉淀”的过程性数据需求。

\section{需求分析}

\subsection{功能性需求}
结合平台定位与多课程适配目标,系统核心功能需求如下:
\begin{enumerate}[label=(\arabic*)]
  \item \textbf{对话式辅导与多模式能力切换}:支持概念讲解、作业反馈、引导式学习、个性化辅导策略生成等模式,并可在课程间复用通用能力。
  \item \textbf{可追溯检索增强}:支持将课程讲义、作业规范与示例等构建为证据库,回答需携带引用编号,支持教师复核与溯源审计。
  \item \textbf{可验证执行能力}:对符号推导、数值计算、仿真与写作规则检查等任务,支持工具调用并回注结果,形成可复核的“计算/校验链”。
  \item \textbf{引导式学习与进度推进}:围绕学习主题生成学习路径并分步提问推进,支持会话恢复与步骤状态更新。
  \item \textbf{学习画像与长期状态追踪}:沉淀薄弱点、完成主题、学习时长等信息,形成课程画像与跨课程全局画像,并支持学习时间线查询。
  \item \textbf{训练与评测闭环}:支持数据规范化、数据蒸馏与 smoke 门禁、LoRA/QLoRA 微调与离线回归评测,为能力迭代提供可复现链路。
\end{enumerate}

\subsection{非功能性需求}
为保证系统在真实教学环境中的可用性与可治理性,非功能需求包括:
\begin{enumerate}[label=(\arabic*)]
  \item \textbf{可信与可复核}:关键结论需能以“引用证据 + 工具结果”进行复核,降低幻觉影响。
  \item \textbf{安全与权限}:基于 RBAC 的角色权限控制,支持课程维度的数据隔离与访问控制;工具调用需具备安全约束,避免越权与资源滥用。
  \item \textbf{可扩展与可维护}:能力扩展应以模块化方式接入(新增技能/工具/课程材料),避免改动核心接口引发兼容性风险。
  \item \textbf{性能与稳定性}:在企业微信 WebView/浏览器等多终端环境中保持稳定交互;检索与工具执行需具备超时与降级策略。
  \item \textbf{隐私与学术诚信}:对学生数据与作业内容进行必要保护;对代写等学术不端请求具备拒答/引导策略。
\end{enumerate}

\section{总体架构设计}

\subsection{分层与服务划分}
系统采用前后端分离与服务化架构,整体分为三层:
\begin{enumerate}[label=(\arabic*)]
  \item \textbf{Web/H5 客户端}:基于 React 构建交互界面,负责对话、学习进度展示、写作分析结果呈现与教师侧学情概览等。
  \item \textbf{业务后端}:基于 Go + Gin 提供统一鉴权、RBAC 权限治理、课程与作业管理、学习画像/事件接口等业务能力。
  \item \textbf{AI 服务层}:基于 FastAPI 提供 OpenAI-compatible 的对话接口与可插拔能力,包括 Skills、GraphRAG 检索增强、Tool Calling 执行器以及引导式学习会话管理等。
\end{enumerate}
该分层使“教学业务逻辑”与“模型推理与能力编排”解耦:后端负责权限、数据与流程治理,AI 服务负责生成与可验证能力落地,前端负责交互与可解释呈现。

\subsection{关键业务流程}
以一次“学生提问/辅导”为例,核心链路可概括为:前端提交问题与上下文 $\rightarrow$ 后端鉴权与权限校验 $\rightarrow$ AI 服务选择技能(\texttt{mode})并可选启用 GraphRAG(\texttt{\_rag} 后缀) $\rightarrow$ 若需要可验证结果则触发工具调用 $\rightarrow$ 回注工具结果与引用证据 $\rightarrow$ 生成最终回答并返回 $\rightarrow$ 后端记录学习事件与画像更新。该流程将生成式回答纳入“可追溯(证据)+ 可验证(工具)+ 可沉淀(画像/事件)”的数据闭环。

\section{模块划分与接口约定}
为便于后续章节的实现描述与复现,本节给出平台关键能力与模块化落地的对应关系:
\begin{table}[htbp]
  \centering
  \caption{关键能力与实现模块对应关系}
  \begin{tabular}{p{3.2cm}p{9.2cm}}
    \toprule
    能力模块 & 实现要点(模块化落地) \\
    \midrule
    Skills(提示编排) & 技能注册表统一管理系统提示与约束,\texttt{mode} 选择技能并支持别名映射;技能可注入课程/作业/画像等上下文。 \\
    GraphRAG(可追溯检索) & 基于图结构索引进行“种子检索 + 图扩展”,回答强制引用编号;支持混合检索与索引更新与 ACL 过滤。 \\
    Tool Calling(可验证执行) & 统一工具参数 Schema 与白名单,限制最大调用次数;工具执行结果回注对话上下文用于最终解释。 \\
    Guided Learning(过程性辅导) & 学习路径生成 + 分步提问推进;会话状态(TTL、用户绑定)维护进度与薄弱点。 \\
    训练与评测 & chat-style 数据规范、LoRA/QLoRA 微调、离线评测报告;训练前增加蒸馏与 smoke 自检门禁。 \\
    学习状态追踪 & 学习事件流 + 课程画像/全局画像;将薄弱点、完成主题、学习时长等沉淀为可追踪档案。 \\
    \bottomrule
  \end{tabular}
\end{table}

接口层面,平台对外保持稳定的对话协议:以 \texttt{messages} 表示多轮对话,以 \texttt{mode} 表示技能选择;当模式名以 \texttt{\_rag} 结尾时,表示在保持技能风格不变的同时启用检索增强。对于需要工具调用的场景,采用结构化工具列表与 \texttt{tool} 回注消息组织对话,使“调用—执行—解释”链路可复现。

\section{数据与权限模型概述}
系统采用 JWT + RBAC 进行鉴权与授权:学生、教师、助教在课程、作业、学习画像等资源上具备不同权限边界。为支持长期学习状态追踪,后端提供课程画像、全局画像与学习事件接口:课程画像用于呈现某门课内的薄弱点与学习进度分布,全局画像用于跨课程归纳学生能力结构,学习事件用于形成可回放的时间线。对于知识库内容,系统在索引与检索环节引入课程维度的访问控制(ACL),避免跨课程材料泄露与越权引用。

\section{本章小结}
本章从应用场景出发给出系统需求与总体设计,明确平台需同时满足多模式辅导、可追溯检索、可验证执行、引导式学习、长期学习追踪以及训练评测闭环等能力,并给出分层架构与模块划分。下一章将进一步围绕 Skills、GraphRAG、工具调用、学习状态追踪与训练评测管线等关键模块展开实现细节。

\chapter{系统关键模块设计与实现}

\section{技能系统与对话编排}
平台将"提示模板、上下文注入、输出约束、安全规则"封装为可复用的技能单元,并通过注册表集中管理。运行时以 \texttt{mode} 参数选择技能,解决了"提示词散落、行为不可控"的工程问题,使系统具备可维护的提示治理能力。

\subsection{技能注册与路由}
AI 服务层将技能定义为一组可调用的能力模块,并为历史模式名提供别名映射以保持接口兼容。平台同时支持 \texttt{\_rag} 后缀约定:当 \texttt{mode} 以 \texttt{\_rag} 结尾时,在保持技能教学风格不变的前提下启用 GraphRAG 检索增强,实现"技能(风格)—检索(证据)"的解耦组合。

\subsection{上下文注入与输出治理}
为使回答贴合课程语境且便于前端结构化展示,技能系统在系统提示中约定输出组织方式(例如先给出结论或评价,再展开解释与建议),并在不干扰用户对话的前提下注入必要上下文:课程信息、作业要求、学生档案与检索证据等。对于存在学术不端风险的请求(如要求代写整篇),系统提示引导模型采取拒答或提供提纲与修改建议等稳健策略,将学术诚信约束纳入可治理范围。

\subsection{已实现技能与能力边界}
从工程落地角度,平台将技能分为"通用能力"与"课程专属能力"两类:
\begin{enumerate}[label=(\arabic*)]
  \item \textbf{通用能力}:概念讲解、作业反馈、引导式学习与个性化辅导策略等,强调结构化输出与过程性引导,可迁移到不同课程。
  \item \textbf{课程专属能力}:以电磁场为例的公式推导与仿真解读;以写作为例的写作类型感知评估与学术规范反馈。课程专属能力通过权限与开关边界控制,避免跨课程误用。
\end{enumerate}
该划分使平台在扩展课程时优先复用通用能力,仅在必要处以课程模块补齐差异化需求。

\section{GraphRAG 检索增强实现}
平台在 AI 服务层实现轻量化 GraphRAG:离线阶段将课程材料切片并构建图结构索引;在线阶段通过混合检索与图扩展获取相关证据片段,并将其注入系统消息,要求回答按引用编号输出。相较传统 RAG,该方案在概念依赖关系较强的任务中更容易覆盖前置概念与相关规范;同时,引用编号为教师复核与错误定位提供了直接抓手。检索阶段支持按课程维度的访问控制,保证不同课程材料隔离。

\section{工具调用执行器与安全治理}

\subsection{调用流程与消息组织}
平台在对话接口中提供工具调用能力:当模型判断需要可验证结果时,生成结构化 \texttt{tool\_calls};系统解析后执行工具,并以 \texttt{tool} 消息将执行结果回注到对话上下文;随后模型基于工具结果生成最终解释与建议。该"调用—执行—回注—解释"链路使关键数值可复现、可审计。

\subsection{安全约束与可控性}
为避免模型进入无限调用或越权调用状态,平台在工程上引入以下约束:
\begin{enumerate}[label=(\arabic*)]
  \item \textbf{工具白名单}:仅允许调用预先注册的工具,未注册名称直接拒绝执行。
  \item \textbf{最大调用次数}:每次对话限制最大工具调用轮数。
  \item \textbf{超时与降级}:工具执行设置超时;不可用时返回失败信息并引导模型给出替代方案。
  \item \textbf{结果回注}:工具输出以结构化文本或 JSON 形式回注,保证后续回答引用的是可复现的外部结果。
\end{enumerate}

\section{引导式学习与学习状态追踪}
平台将引导式学习作为"学生中心"的核心交互方式:系统围绕学习主题生成学习路径,并在多轮对话中维护进度、薄弱点与阶段性目标,实现"诊断—引导—巩固"的过程性辅导。

\subsection{学习路径生成与会话管理}
引导式学习首先生成结构化学习路径(3--6 步),并为每一步附带目标描述、前置概念与是否需要工具验证等标记。系统以 \texttt{session\_id} 绑定会话与用户身份,并引入有效期与会话数量上限,避免无界增长与越权访问。会话推进过程中,系统维护学习目标、当前步骤、历史对话与步骤完成状态。

\subsection{薄弱点检测与学习画像}
平台对辅导过程中的"纠错/提示"语句进行轻量解析,在负面信号上下文中提取概念薄弱点(如写作中的"逻辑连接"或理工科中的"边界条件"),并在会话内累积统计。后端提供学习画像与学习事件接口:将薄弱点、完成主题与学习时长沉淀为课程画像与跨课程全局画像,并以学习事件流形成可回放时间线,为教师侧学情分析提供数据支撑。

\subsection{个性化辅导策略生成}
在具备学习档案后,系统可基于历史薄弱点与学习进度生成结构化辅导策略,例如 1--2 周学习计划、针对薄弱点的专项练习与推荐主题。该能力同样通过技能系统实现,输出结构化数据便于前端展示与教师复核。

\section{异构加速服务部署}
AI 服务通过 OpenAI-compatible 接口与上游推理服务对接,部署侧支持 GPU、NPU 与 FPGA 三种异构加速后端,分别针对大模型推理与向量检索进行优化。

\subsection{GPU 推理部署}
GPU 方案采用 vLLM 作为推理引擎,加载 Hugging Face 格式模型,支持连续批处理与 PagedAttention 以提升吞吐。部署时通过 Docker 容器化,配置 CUDA 驱动与显存限制即可启动。该方案在 NVIDIA RTX 4090 上可获得较低延迟与较高吞吐。

\subsection{Ascend NPU 推理部署}
为满足国产化与节能需求,平台同时支持华为 Ascend NPU 后端。部署流程如下:
\begin{enumerate}[label=(\arabic*)]
  \item \textbf{模型转换}:使用 ATC 将 Hugging Face 模型转换为 Ascend 可执行格式(OM 模型),或使用 MindIE 直接加载权重。
  \item \textbf{推理引擎}:采用 MindIE 提供 Python/C++ 推理接口,或通过 vLLM 的 Ascend 后端实现兼容。
  \item \textbf{服务封装}:推理服务对外暴露 OpenAI-compatible 接口,AI 服务层无需区分上游是 GPU 还是 NPU 后端。
\end{enumerate}
实测表明,Ascend 910B 在首 token 延迟上与 RTX 4090 处于同一数量级,后续 token 吞吐略低约 15\%,但满载功耗降低约 20\%。

\subsection{FPGA 加速 Embedding 服务}
在检索增强链路中,Embedding 计算是主要延迟来源之一。当 GPU 被大模型推理占用时,Embedding 若共享 GPU 会产生资源竞争。平台引入 FPGA 加速方案,将 Embedding 计算迁移至 Xilinx Alveo 加速卡,实现资源解耦与延迟优化。

\textbf{实现流程}:
\begin{enumerate}[label=(\arabic*)]
  \item \textbf{模型量化}:将 Embedding 模型(如 bge-base-zh、sentence-transformers)通过 Vitis AI 量化工具转换为 INT8 精度。量化过程使用课程语料库的代表性样本进行校准,保证量化后的向量表示与 FP32 版本保持较高余弦相似度($>0.98$)。
  \item \textbf{模型编译}:使用 Vitis AI Compiler 将量化模型编译为 FPGA 可执行指令(xmodel),指定目标设备(如 U50、U250)的 DPU 配置。
  \item \textbf{硬件部署}:在 Xilinx Alveo 加速卡上部署 xmodel,通过 DPU(Deep Learning Processing Unit)IP 核执行推理。FPGA 通过 PCIE 与主机通信,使用 VART(Vitis AI Runtime)API 进行调用。
  \item \textbf{服务封装}:将 FPGA Embedding 推理封装为 FastAPI 微服务,对外提供与 CPU/GPU Embedding 服务相同的 REST 接口(\texttt{/v1/embeddings}),便于 AI 服务层无感切换。
\end{enumerate}

\textbf{接口设计}:
\begin{verbatim}
POST /v1/embeddings
{
  "model": "bge-base-zh-fpga",
  "input": ["query text 1", "query text 2"]
}
Response:
{
  "data": [
    {"embedding": [0.1, 0.2, ...], "index": 0},
    {"embedding": [0.3, 0.4, ...], "index": 1}
  ]
}
\end{verbatim}

\textbf{配置切换}:AI 服务通过环境变量 \texttt{EMBEDDING\_BACKEND} 选择后端(\texttt{cpu}、\texttt{gpu}、\texttt{fpga}),无需修改业务代码即可切换。

\section{训练、蒸馏与评测管线}
为提升模型在教学风格、引用规范、工具调用与引导式学习等能力上的稳定性,平台提供可复现的训练与评测工具链。

\subsection{数据规范与样本类型}
训练数据采用 chat-style JSONL 表示:每条样本包含 \texttt{mode} 与多轮 \texttt{messages},与运行时接口保持一致。数据按任务属性分桶组织,典型包括:
\begin{enumerate}[label=(\arabic*)]
  \item \textbf{教学风格与结构化输出}:约束"先结论—再解释—再建议"的组织方式,保证格式可解析。
  \item \textbf{RAG 引用约束}:注入证据片段并要求仅基于片段作答,强制标注引用编号,证据不足时追问或拒答。
  \item \textbf{工具调用样本}:包含工具调用请求与结果回注,使模型学会在需要可验证结果时触发工具并正确解释。
  \item \textbf{拒答与追问}:对缺参、超范围或学术不端请求给出稳健行为。
\end{enumerate}
数据规范化的直接收益是训练与上线协议对齐,减少格式漂移导致的线上不稳定。

\subsection{LoRA/QLoRA 微调与产物管理}
平台采用 LoRA/QLoRA\cite{hu2021lora,dettmers2023qlora} 对基座模型进行参数高效微调,并将产物以 adapter 形式输出。训练侧同时输出 adapter 权重、训练配置与日志;支持训练完成后自动生成预测与评测报告,形成可对比的版本迭代依据。

\subsection{数据蒸馏与质量门禁}
在启动微调前,平台引入"蒸馏 + smoke"作为前置门禁:将 chat-style 数据蒸馏为易于审阅的格式并统计去重率;随后以分钟级轻量指标验证数据链路可用,用于发现空样本、重复激增或字段缺失等问题。smoke 指标仅用于链路检查,不用于宣称最终效果。

\section{课程示例模块}

\subsection{电磁场:数值仿真与推导验证}
对电磁场类推导与计算任务,平台可扩展数值仿真与可视化能力,并通过工具调用将关键计算交由可执行组件完成。例如,采用有限差分法求解二维静电场(Laplace/Poisson 方程)或同轴线电容场分布,输出字段数据与可视化图像,再由模型结合引用证据与仿真结果进行解释与引导,形成"计算—解释—理解"的教学闭环。

\subsection{研究生专业英文写作:结构化评价与规范反馈}
对写作类课程,系统以结构化评价维度(rubrics)组织反馈,强调问题定位、修改顺序与可执行建议;对引用规范、段落结构与格式等硬约束项,可交由工具进行规则化检查,减少凭空建议的风险。技能提示中显式加入学术诚信约束,默认不代写整篇内容,而以示范片段、段落框架与改写建议辅助学生完成迭代。

\section{本章小结}
本章围绕平台工程落地给出关键模块的设计与实现:以技能系统进行能力编排与治理;以 GraphRAG 提供可追溯证据链;以工具调用提供可验证执行链;以引导式学习与学习画像实现过程性辅导与长期追踪;异构加速层支持 GPU、NPU 与 FPGA 三种后端——GPU/NPU 用于大模型推理,FPGA 用于 Embedding 服务的低延迟加速与资源解耦;训练、蒸馏与评测管线支撑能力迭代。下一章将从测试与实验评估角度验证系统方案与实现效果。

\chapter{测试与实验评估}

\section{评估维度与指标设计}
本章围绕四个维度对平台进行评估:功能正确性验证系统各模块能否按预期工作;可追溯性验证答案能否溯源至检索证据;可验证性验证工具调用结果是否被正确引用;可迭代性验证回归评测能否有效约束迭代风险。具体评估指标包括:
\begin{enumerate}[label=(\arabic*)]
  \item \textbf{功能正确性}:鉴权与权限边界是否正确、核心业务流程能否正常运行。
  \item \textbf{引用一致性}:答案中的引用编号是否与检索片段一一对应;证据不足时系统是否触发追问或拒答。
  \item \textbf{工具调用可用性}:模型是否在合适时机触发调用、参数能否正确解析、执行结果是否被准确解释。
  \item \textbf{结构化输出稳定性}:JSON 或分点列表是否可解析,便于前端渲染与教师复核。
  \item \textbf{学习追踪完整性}:会话状态能否正确恢复、薄弱点与学习时长能否持续沉淀。
\end{enumerate}

\section{测试环境配置}

\subsection{硬件环境}
表~\ref{tab:hw-env}~列出了测试所用硬件。为对比 GPU、NPU 与 FPGA 三种异构加速后端,分别部署对应推理服务。

\begin{table}[htbp]
  \centering
  \caption{测试硬件配置}
  \label{tab:hw-env}
  \begin{tabular}{llll}
    \toprule
    配置项 & GPU 环境 & NPU 环境 & FPGA 环境 \\
    \midrule
    服务器 CPU & Intel Xeon 8核 & 鲲鹏 920 64核 & Intel Xeon 8核 \\
    内存 & 32 GB & 64 GB & 32 GB \\
    加速卡 & RTX 4090 24GB & Ascend 910B 64GB & Xilinx Alveo U50 \\
    加速卡功耗 & 350 W(满载) & 280 W(满载) & 75 W(满载) \\
    存储 & NVMe SSD 512 GB & NVMe SSD 1 TB & NVMe SSD 512 GB \\
    \bottomrule
  \end{tabular}
\end{table}

\subsection{软件环境}
\begin{table}[htbp]
  \centering
  \caption{测试软件配置}
  \begin{tabular}{llll}
    \toprule
    组件 & GPU 环境 & NPU 环境 & FPGA 环境 \\
    \midrule
    操作系统 & Ubuntu 22.04 & openEuler 22.03 & Ubuntu 22.04 \\
    LLM 推理框架 & vLLM 0.4.x & MindIE 1.0 & — \\
    Embedding 框架 & sentence-transformers & sentence-transformers & Vitis AI 3.5 \\
    Python & 3.11 & 3.9 & 3.10 \\
    基座模型 & Qwen2.5-7B-Instruct & Qwen2.5-7B-Instruct & — \\
    Embedding 模型 & bge-base-zh (FP32) & bge-base-zh (FP32) & bge-base-zh (INT8) \\
    \bottomrule
  \end{tabular}
\end{table}

\section{功能测试}

\subsection{核心功能用例}
表~\ref{tab:test-cases}~汇总了各模块的功能测试结果。测试覆盖用户鉴权、权限控制、对话服务、检索增强、工具调用、引导式学习、学习画像与写作分析等关键链路。

\begin{table}[htbp]
  \centering
  \caption{核心功能测试结果}
  \label{tab:test-cases}
  \begin{tabular}{p{2.5cm}p{4.5cm}p{2cm}p{3cm}}
    \toprule
    模块 & 测试内容 & 预期 & 结果 \\
    \midrule
    用户鉴权 & 持有效 JWT 访问受保护接口 & 200 OK & 通过 \\
    权限控制 & 学生角色访问教师接口 & 403 Forbidden & 通过 \\
    对话服务 & tutor 模式概念解释 & 结构化回答 & 通过 \\
    检索增强 & tutor\_rag 模式带引用回答 & 引用编号正确 & 通过 \\
    工具调用 & 数值计算触发工具 & 返回可验证结果 & 通过 \\
    引导式学习 & 创建会话并推进步骤 & 状态更新 & 通过 \\
    学习画像 & 查询课程画像接口 & 返回薄弱点列表 & 通过 \\
    写作分析 & 提交摘要获取反馈 & 多维度评分 & 通过 \\
    \bottomrule
  \end{tabular}
\end{table}

\section{异构加速性能对比}
本节对比 GPU、NPU 与 FPGA 三种后端在大模型推理与 Embedding 计算两个环节的性能表现。

\subsection{大模型推理:GPU vs NPU}
测试任务为单轮对话,输入 prompt 约 500 token,输出约 200 token。

\begin{table}[htbp]
  \centering
  \caption{GPU 与 NPU 大模型推理性能对比}
  \label{tab:gpu-npu}
  \begin{tabular}{lrr}
    \toprule
    指标 & RTX 4090 & Ascend 910B \\
    \midrule
    首 token 延迟(TTFT) & 180 ms & 210 ms \\
    后续 token 吞吐 & 42 token/s & 36 token/s \\
    单请求端到端延迟 & 4.9 s & 5.6 s \\
    满载功耗 & 350 W & 280 W \\
    \bottomrule
  \end{tabular}
\end{table}

Ascend 910B 的首 token 延迟与 RTX 4090 处于同一数量级,吞吐略低约 15\%,但功耗降低约 20\%,适合国产化与绿色部署场景。

\subsection{Embedding 计算:CPU vs GPU vs FPGA}
为评估将 Embedding 计算迁移至 FPGA 后的性能影响,设计以下对照组:

\begin{table}[htbp]
  \centering
  \caption{Embedding 后端对照组设计}
  \begin{tabular}{llll}
    \toprule
    组别 & Embedding 执行位置 & 精度 & 备注 \\
    \midrule
    G0(基线) & CPU(sentence-transformers) & FP32 & 最小可用方案 \\
    G1 & GPU(sentence-transformers) & FP32 & 观察 GPU 竞争 \\
    G2(目标) & FPGA(Vitis AI) & INT8 & 仅替换 Embedding \\
    \bottomrule
  \end{tabular}
\end{table}

测试任务为批量 Embedding 计算,batch size = 8,文本长度约 128 token。

\begin{table}[htbp]
  \centering
  \caption{Embedding 后端性能对比}
  \label{tab:embed-perf}
  \begin{tabular}{lrrrr}
    \toprule
    组别 & P50 延迟(ms) & P95 延迟(ms) & 吞吐(QPS) & 功耗(W) \\
    \midrule
    G0(CPU) & 85 & 120 & 12 & 65 \\
    G1(GPU) & 15 & 25 & 65 & 180 \\
    G2(FPGA) & 18 & 28 & 55 & 45 \\
    \bottomrule
  \end{tabular}
\end{table}

\textbf{分析}:
\begin{enumerate}[label=(\arabic*)]
  \item FPGA 方案(G2)的延迟与 GPU(G1)处于同一数量级,P95 延迟仅高约 12\%。
  \item FPGA 吞吐略低于 GPU(约 85\%),但功耗仅为 GPU 的 25\%,能效比显著更优。
  \item 将 Embedding 从 GPU 迁移至 FPGA 后,GPU 可专注于大模型推理,整体并发能力提升。
\end{enumerate}

\subsection{端到端 RAG 链路对比}
将 Embedding 后端切换后,评估完整 RAG 链路的端到端延迟变化。测试任务为带检索增强的写作规范问答,top\_k = 5。

\begin{table}[htbp]
  \centering
  \caption{RAG 端到端延迟对比}
  \label{tab:rag-e2e}
  \begin{tabular}{lrrr}
    \toprule
    组别 & T\_embed(ms) & T\_rag\_total(ms) & T\_e2e(s) \\
    \midrule
    G0(CPU Embed) & 85 & 180 & 5.2 \\
    G1(GPU Embed) & 15 & 110 & 4.9 \\
    G2(FPGA Embed) & 18 & 115 & 4.95 \\
    \bottomrule
  \end{tabular}
\end{table}

FPGA 方案的端到端延迟与 GPU 方案基本持平,瓶颈已从 Embedding 转移到 LLM 生成阶段。

\section{检索质量评估}
为验证 FPGA 量化(INT8)是否影响检索质量,使用固定回归集对三种 Embedding 后端进行离线评测。

\subsection{回归集构成}
回归集按写作类型分层采样,共 200 条查询:
\begin{table}[htbp]
  \centering
  \caption{检索质量评测回归集}
  \begin{tabular}{lrl}
    \toprule
    类型 & 样本数 & 说明 \\
    \midrule
    文献综述(literature\_review) & 50 & 需引用规范片段 \\
    课程论文(course\_paper) & 50 & 需引用范例 \\
    学位论文章节(thesis) & 50 & 需引用结构规范 \\
    摘要(abstract) & 50 & 含证据不足样本 \\
    \midrule
    合计 & 200 & — \\
    \bottomrule
  \end{tabular}
\end{table}

\subsection{质量指标结果}
\begin{table}[htbp]
  \centering
  \caption{检索质量指标对比}
  \label{tab:quality}
  \begin{tabular}{lrrrr}
    \toprule
    组别 & Recall@5 & 引用一致性 & 拒答准确率 & 向量余弦相似度 \\
    \midrule
    G0(CPU FP32) & 88.5\% & 92.5\% & 85.0\% & — \\
    G1(GPU FP32) & 88.5\% & 92.5\% & 85.0\% & — \\
    G2(FPGA INT8) & 87.0\% & 91.0\% & 84.0\% & 0.985 \\
    \bottomrule
  \end{tabular}
\end{table}

\textbf{分析}:
\begin{enumerate}[label=(\arabic*)]
  \item FPGA INT8 量化后,Recall@5 下降约 1.5 个百分点,引用一致性下降约 1.5 个百分点,在可接受范围内。
  \item INT8 向量与 FP32 向量的平均余弦相似度为 0.985,表明量化对语义表示影响较小。
  \item 拒答准确率基本持平,证据不足时的稳健行为未受量化影响。
\end{enumerate}

\section{离线评测与回归机制}
为支撑模型与提示策略的持续迭代,系统采用"固定回归集 + 指标报告 + 案例分析"的评测范式。

\subsection{回归集构成}
回归集按任务类型组织,当前规模 120 条,覆盖平台主要能力:

\begin{table}[htbp]
  \centering
  \caption{模型能力回归集构成}
  \begin{tabular}{lrl}
    \toprule
    任务类型 & 样本数 & 评测重点 \\
    \midrule
    tutor(概念讲解) & 30 & 结构化输出、教学风格 \\
    grader(作业反馈) & 25 & 评价维度、可执行建议 \\
    guided\_learning & 20 & 路径合理性、步骤推进 \\
    tool\_calling & 15 & 调用时机、参数正确性 \\
    rag\_citation & 20 & 引用一致性、证据相关性 \\
    refusal/followup & 10 & 拒答与追问稳健性 \\
    \midrule
    合计 & 120 & — \\
    \bottomrule
  \end{tabular}
\end{table}

\subsection{评测指标}
表~\ref{tab:eval-metrics}~给出当前版本在回归集上的各项指标。

\begin{table}[htbp]
  \centering
  \caption{离线评测指标}
  \label{tab:eval-metrics}
  \begin{tabular}{lrl}
    \toprule
    指标 & 数值 & 说明 \\
    \midrule
    引用一致性 & 92.5\% & 引用编号与检索片段对应 \\
    工具调用准确率 & 88.0\% & 调用时机正确且参数可解析 \\
    结构化输出可解析率 & 95.0\% & JSON/分点格式正确 \\
    拒答/追问正确率 & 85.0\% & 证据不足时采取稳健策略 \\
    教学风格符合度 & 90.0\% & 人工评分 \\
    \bottomrule
  \end{tabular}
\end{table}

\section{典型场景案例}

\subsection{电磁场场景}
以静电场边界条件为例,展示系统的"证据链 + 计算链"输出方式。

\textbf{学生提问}:请解释平面电容器两介质分界面上的电场边界条件。

\textbf{系统检索}:检索到 3 个相关片段——[1] 边界条件定义、[2] 法向分量连续性、[3] 切向分量连续性。

\textbf{系统回答}(节选):
\begin{quote}
在两种介质接触面上,电场需满足以下条件:

1. 法向分量:$D_{1n} = D_{2n}$(无自由面电荷时)\,[1][2]

2. 切向分量:$E_{1t} = E_{2t}$\,[3]

这表明……(后文省略)
\end{quote}

上述回答中,引用编号 [1][2][3] 与检索片段一一对应,结论基于证据生成,未出现凭空编造现象。

\subsection{写作场景}
以论文摘要评估为例,展示结构化反馈能力。

\textbf{学生提交}:一段约 200 词的英文摘要草稿。

\textbf{系统反馈}(节选):
\begin{quote}
\textbf{写作类型}:摘要(Abstract)

\textbf{综合评分}:72/100

\textbf{分维度评估}:
\begin{itemize}
  \item 学术语气:7/10 —— 建议将 ``really important'' 改为 ``significant''
  \item 段落结构:8/10 —— 背景-方法-结果-结论框架完整
  \item 逻辑连接:6/10 —— 第 2--3 句之间缺少过渡
  \item 引用规范:N/A(摘要通常不含引用)
\end{itemize}

\textbf{修改优先级}:1.\,逻辑连接 → 2.\,学术语气
\end{quote}

该反馈按评估框架输出,建议具有可操作性,便于学生按优先级逐项修改。

\section{本章小结}
本章从功能测试、性能测试与离线评测三个层面对平台进行验证。功能测试覆盖鉴权、权限、对话、检索增强、工具调用、引导式学习、学习画像与写作分析等关键链路,全部通过。异构加速性能测试对比了 GPU(RTX 4090)、NPU(Ascend 910B)与 FPGA(Alveo U50)三种后端:NPU 在大模型推理中首 token 延迟与 GPU 接近,功耗降低 20\%;FPGA 在 Embedding 计算中延迟与 GPU 持平,功耗仅为 GPU 的 25\%,且将 Embedding 从 GPU 解耦后可提升整体并发能力。检索质量评估表明,FPGA INT8 量化对 Recall@5 的影响约 1.5 个百分点,处于可接受范围。离线评测在回归集上验证了引用一致性 92.5\%、工具调用准确率 88.0\% 等指标,表明系统在可追溯与可验证方面达到预期目标。

\chapter{总结与展望}

\section{工作回顾}
本文针对多课程教学场景中"反馈成本较高、学习过程难以追踪、生成式模型可信度不足"等问题,设计并实现了一套以学生为中心的智能教学平台。下面从四个方面对所做工作进行回顾。

\subsection{系统架构与工程实现}
平台采用前后端分离与服务化架构:前端以 React + TypeScript + Vite 构建 Web/H5 客户端,可嵌入企业微信 WebView 并保持跨端一致体验;后端基于 Go + Gin 提供业务 API,使用 JWT 进行无状态认证,配合 RBAC 模型实现多角色权限管理;AI 服务以 Python + FastAPI 实现,通过 OpenAI-compatible 接口与上游推理服务对接。代码采用 Monorepo 组织,前后端共享类型定义,降低接口不一致风险。在异构加速方面,系统支持 GPU、NPU 与 FPGA 三种后端——GPU/NPU 用于大模型推理,FPGA 用于 Embedding 服务的低延迟加速与资源解耦。

\subsection{可追溯与可验证的智能辅导}
为应对大语言模型的幻觉与可信度问题,平台构建了"证据链 + 计算链"的双重可追溯机制:
\begin{enumerate}[label=(\arabic*)]
  \item \textbf{GraphRAG 检索增强}:将课程讲义与规范构建为图结构索引,采用混合检索与图扩展获取相关证据,回答中以编号形式标注来源,使结论可追溯。
  \item \textbf{工具调用}:对数值计算、仿真与格式检查等任务,通过工具调用获取可执行结果,降低模型"心算失误"与"凭空建议"的风险。
\end{enumerate}

\subsection{过程性辅导与学习追踪}
平台以引导式学习为核心交互方式,将复杂学习主题拆解为可管理的步骤,逐步提问推进,实现"诊断—引导—巩固"的过程性辅导。同时,通过学习事件流与多级学习画像(课程画像/全局画像)将薄弱点、学习时长与完成主题等信息沉淀为可查阅档案,为教师侧学情分析与个性化干预提供数据支撑。

\subsection{训练与评测链路}
为支持模型定制与持续迭代,平台提供完整的后训练工具链:数据规范化确保训练与上线协议一致;数据蒸馏与 smoke 验证作为质量门禁;LoRA/QLoRA 参数高效微调输出可版本化的 adapter;离线回归评测通过固定评测集量化引用一致性、工具调用准确率与结构化输出稳定性等指标。

\section{主要贡献}
本文的主要贡献可归纳为以下五点:
\begin{enumerate}[label=(\arabic*)]
  \item 提出并落地"学生中心数据闭环 + GraphRAG 可追溯 + 工具调用可验证 + 引导式学习"的一体化框架,形成可复用的工程路径。
  \item 设计"通用能力 + 课程专属模块"的模块化策略,以电磁场与研究生专业英文写作为例验证跨课程可迁移性。
  \item 对比 GPU 与 NPU 两种大模型推理后端的性能与能耗,为校园场景下的国产化与绿色部署提供参考。
  \item 引入 FPGA 加速 Embedding 服务,实现检索链路与大模型推理的资源解耦,在保持检索质量的前提下显著降低能耗。
  \item 建立面向持续迭代的训练与回归评测流程,使模型迭代从主观判断转向可量化回归。
\end{enumerate}

\section{不足与改进方向}
尽管本文完成了平台原型实现,仍存在以下局限:

\subsection{当前不足}
\begin{enumerate}[label=(\arabic*)]
  \item \textbf{知识图谱粒度}:当前 GraphRAG 以文档片段为节点,对概念、公式与物理量之间细粒度关系的建模仍有欠缺。
  \item \textbf{评测集规模}:回归集约百条量级,难以覆盖所有边界情况;需进一步扩展并引入人工评测。
  \item \textbf{真实场景验证}:当前验证主要基于模拟数据,尚未在真实课程中大规模部署并收集师生反馈。
  \item \textbf{企业微信深度集成}:OAuth 与消息推送能力处于预留状态,未完成完整的内嵌体验。
  \item \textbf{FPGA 量化损失}:INT8 量化导致 Recall@5 下降约 1.5 个百分点,对高精度场景需进一步优化量化策略。
\end{enumerate}

\subsection{后续工作}
针对上述不足,可从以下方向继续改进:
\begin{enumerate}[label=(\arabic*)]
  \item \textbf{知识抽取与融合}:引入实体关系抽取与图数据库(如 Neo4j),提升知识图谱的细粒度与可维护性。
  \item \textbf{评测体系扩展}:构建千条量级的课程问答评测集,量化准确率与教学效果;引入 A/B 测试框架。
  \item \textbf{检索与上下文优化}:优化检索排序与上下文压缩策略,降低延迟并提升长上下文任务的稳定性。
  \item \textbf{企业微信完整集成}:完成 OAuth 认证与消息推送能力,提升触达体验。
  \item \textbf{多模态能力}:引入图像理解,支持手写作业识别与公式图片解析。
  \item \textbf{学习分析可视化}:为教师提供班级学情看板,展示薄弱点分布与能力发展趋势。
  \item \textbf{异构加速生态完善}:跟进 Ascend 与 Xilinx 软件栈更新,探索 Reranker 的 FPGA 加速与混合精度量化。
\end{enumerate}

\section{结语}
本文完成了一套以学生为中心的智能教学平台原型,在"可追溯、可验证、可迭代"的设计目标下,为将大语言模型可控地引入教育场景提供了工程实践参考。通过引入 GPU、NPU 与 FPGA 三种异构加速后端,平台在性能、能耗与资源利用之间取得了更好的平衡。希望本研究能够为高校智能教学系统的建设提供借鉴,并在后续工作中不断完善与推广。


\appendix
\chapter{附录}
在此放置补充材料,例如证明、额外实验结果或关键代码片段。


\backmatter
\printbibliography[heading=bibintoc]

\end{document}
