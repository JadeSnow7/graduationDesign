\chapter{绪论}
\section{研究背景与意义}
《电磁场与电磁波》课程是电子信息类、通信工程等专业的核心基础课程,知识点抽象、公式推导多且逻辑链条长,学生在自学与复习过程中容易出现概念混淆与步骤缺失。传统课堂与常规教学平台在即时答疑、过程性引导与知识结构呈现方面能力有限,教师在作业批改、答疑与学情分析中投入大量重复性工作。近年来,Transformer 架构\cite{vaswani2017}推动大语言模型在自然语言理解与生成方面取得突破,对话式模型(如 ChatGPT)\cite{openai2022}显示出辅助教学与答疑的潜力,但其幻觉与可解释性问题仍需外部知识约束。检索增强生成(RAG)\cite{lewis2020}与知识图谱\cite{hogan2020}提供了结构化知识支撑与可追溯机制,为构建面向课程教学的智能问答系统提供了新的技术路径。基于企业微信的 H5 入口具备便捷触达与组织管理优势,适合承载课程智能教学平台原型。

\section{国内外研究现状}
在教育信息化领域,学习管理系统与学习分析工具已实现课程资源管理、作业统计与学习过程记录,但对推导型课程的概念解释与解题引导支持不足。大语言模型在教育问答、自动反馈等场景中应用快速增长,研究重点从“能回答”转向“可追溯、可验证”,RAG 被广泛用于降低幻觉并提升答案依据\cite{lewis2020}。知识图谱作为结构化语义组织方式,在教材知识组织、概念关联与搜索推荐中具有优势\cite{hogan2020},但与大模型深度协同的课程问答系统仍处于探索阶段。综上,面向《电磁场与电磁波》课程构建知识图谱并与大模型协同的问答引擎,具有明确的研究与应用价值。

\section{研究内容与任务要求}
本文围绕课程知识图谱与协同问答引擎的构建开展研究,主要内容与任务要求如下。

\subsection{课题内容}
\begin{enumerate}[label=(\arabic*)]
  \item 《电磁场与电磁波》课程知识图谱构建。
  \item 大语言模型与知识图谱的协同问答引擎设计。
  \item 系统原型实现与企业微信集成或网页实现。
  \item 系统测试与评估。
\end{enumerate}

\subsection{课题任务要求}
\begin{enumerate}[label=(\arabic*)]
  \item 深入理解大语言模型的基本原理及其应用范式,掌握至少一种主流 LLM 的 API 调用方法。
  \item 掌握知识图谱的构建流程,能够针对《电磁场与电磁波》内容进行知识抽取、融合与存储。
  \item 完成一个包含知识图谱管理、智能问答交互集成功能的完整系统原型。
  \item 系统应能准确回答课程相关的概念、公式、定理等基础问题,并能进行简单的习题求解引导。
  \item 完成毕业设计论文的撰写,论文应结构清晰、论证充分、代码和数据详实。
\end{enumerate}

\section{论文结构}
本文共分三章:第一章介绍研究背景、研究现状以及课题内容与任务要求;第二章给出需求分析、系统架构与关键模块设计实现;第三章总结工作并展望后续改进方向。
