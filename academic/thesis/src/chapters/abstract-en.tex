\chapter*{Abstract}
\addcontentsline{toc}{chapter}{Abstract}
This thesis targets graduate-level academic writing instruction, where feedback is costly, student needs are heterogeneous, and progress is difficult to track. We design and implement a student-centric intelligent teaching platform that forms a closed loop: writing submission, automated analysis, personalized feedback, instructor review, and profile updates. The platform is writing-type aware and supports literature reviews, course papers, thesis sections, and abstracts with type-specific rubrics and prompting strategies. We adopt a service-oriented architecture with a Go backend for course and writing workflows, RBAC, and learning-event logging, and a FastAPI AI service that calls an OpenAI-compatible LLM endpoint and exposes chat tutoring, writing analysis, tool calling, and GraphRAG-based retrieval augmentation. To reduce hallucinations and improve traceability, we build a writing knowledge base from course guidelines, exemplars, and common error patterns, retrieve evidence snippets, and require citation markers in generated feedback. For model customization, we provide an end-to-end post-training toolchain across cloud and edge settings, including LoRA/QLoRA pipelines and a local ms-swift LoRA run for Qwen3-0.6B on 2026-02-10 with client integration verification. On the client side, we support both Web and Mobile via shared types and a unified SDK to keep API contracts consistent while adapting UI for each platform. The implementation demonstrates a practical and controllable workflow for academic writing assistance and lays the groundwork for data-driven iterative improvement.

\vspace{0.5cm}
\noindent\textbf{Keywords:} academic writing; student profile; retrieval-augmented generation;\\ GraphRAG; LoRA/QLoRA; cross-platform SDK
\clearpage
