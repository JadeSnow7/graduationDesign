\chapter{方法与实现}
\section{需求分析与总体设计}
平台面向管理员、教师、助教与学生四类角色,核心需求包括课程知识图谱构建与管理、协同问答交互、企业微信/网页访问与教学数据记录。系统需支持概念、公式与定理的准确问答以及简单习题的求解引导,同时强调回答可追溯性与安全性。非功能需求关注响应时间、并发能力、数据安全、可维护性与跨终端兼容性。基于上述需求,系统采用前后端分离与服务化架构,将知识图谱索引与大模型问答能力解耦,通过后端统一鉴权与权限控制,确保能力可扩展与可治理。

\section{系统架构}
系统整体采用“前端—后端—问答服务—知识图谱索引”的四层结构:前端提供企业微信内嵌 H5 与浏览器访问入口;后端提供业务 API、JWT 鉴权与 RBAC 权限控制;问答服务负责提示模板与模型调用;知识图谱以离线构建的图结构索引形式存储与检索。整体架构如图~\ref{fig:architecture} 所示。

\begin{figure}[htbp]
  \centering
  \fbox{\rule{0pt}{5cm}\rule{10cm}{0pt}}
  \caption{平台总体架构示意(占位)}
  \label{fig:architecture}
\end{figure}

\section{知识图谱构建}
知识图谱以课程讲义、教材章节、习题解析与教学文档为主要数据源。构建流程包括数据清洗、章节切分、知识点抽取与结构化组织,并建立概念、公式、定理与实例之间的关联关系。原型阶段采用基于 Markdown 标题层级的轻量抽取方案,将章节作为节点、层级关系作为边,结合正文片段形成可检索的图结构索引,存储为 JSON 文件并支持离线更新。该方案便于快速构建课程级知识图谱,并可扩展到实体关系抽取与图数据库存储。

\section{协同问答引擎设计}
协同问答引擎采用 GraphRAG 的“检索增强 + 大模型生成”流程:用户问题首先映射到知识图谱索引进行检索,得到相关知识片段与节点上下文;随后将检索结果与系统提示模板一起输入大语言模型,生成可追溯的回答。为满足教学场景需求,系统设计了概念解释、公式验证与习题引导等多种模式,要求回答引用检索片段并避免直接给出完整答案。该设计兼顾答疑效果与安全控制,提升回答一致性与可信度。

\subsection{GraphRAG 检索增强}
GraphRAG 将课程知识图谱的节点关系与文本片段索引结合,形成“图结构 + 片段检索”的混合检索机制。检索阶段先召回与问题相关的片段作为种子,再沿知识图谱邻接关系扩展候选片段,最终进行重排序并截断上下文长度。该机制在保证相关性的同时扩大了上下文覆盖度,便于模型在回答时引用节点关系与章节结构,提高回答的可追溯性与一致性。

\section{系统原型实现与企业微信集成}
前端基于 Vue 3 + Vite 实现 H5 页面,兼容企业微信内置浏览器与普通浏览器;后端采用 Go/Gin 实现 API 与权限治理,支持账号密码登录并预留企业微信 OAuth 接入。问答服务基于 FastAPI 实现,对接 OpenAI 兼容接口并内置多种系统提示模板;当请求模式带有 \texttt{\_rag} 后缀时,服务加载 GraphRAG 索引并注入检索片段。原型核心接口包括 \texttt{/api/v1/auth/login}、\texttt{/api/v1/auth/wecom}、\texttt{/api/v1/ai/chat} 等,部署层面提供 Docker Compose 配置以便快速启动与验证。

\section{系统测试与评估}
测试与评估围绕功能正确性、问答准确性与系统性能展开。功能测试验证知识图谱索引构建、问答流程与登录鉴权的完整性;准确性测试可采用人工标注的课程问答集,统计概念解释正确率与引用一致性;性能测试关注典型问题的响应时间与并发稳定性。上述指标可结合真实课程数据进一步补充,形成可量化的评估体系。
