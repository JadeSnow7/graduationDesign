\chapter*{摘要}
\addcontentsline{toc}{chapter}{摘要}
本文面向本科生《电磁场与电磁波》与研究生《学术规范与论文写作》教学中“反馈成本高、个体差异大、过程难追踪”等问题,设计并实现一套以学生为中心的智能教学平台。平台以写作任务为主线,构建“写作提交—智能分析—个性化反馈—教师复核—学习画像更新”的闭环,支持文献综述、课程论文、学位论文与摘要等写作类型的差异化评价与针对性建议。系统采用前后端分离与服务化架构:后端负责课程与写作业务、RBAC 权限控制以及学习事件记录;AI 服务通过 OpenAI-compatible 接口调用大语言模型,提供对话辅导、引导式学习、写作分析、工具调用与 GraphRAG 检索增强能力。为降低幻觉并提升可追溯性,平台针对《学术规范与论文写作》课程,将课程规范、优秀范文片段与常见错误库构建为可检索向量知识库,在生成反馈时注入证据片段并要求引用编号,支持事后审核与复核。为支撑模型定制,本文实现了覆盖云侧与端侧的后训练工具链:在写作场景建立 LoRA/QLoRA 训练评测流程,并在 2026-02-10 完成基于 ms-swift 的端侧 qwen3-0.6B 本机 LoRA 微调与客户端联测闭环。客户端侧采用跨端方案,通过共享 types 与统一 SDK 保持 Web 与移动端的 API 契约一致,并在 UI 层针对不同终端做适配。实践表明,该系统能够以较低的工程成本实现可控、可追溯的写作辅导流程,并为规模化数据驱动的模型迭代提供基础。

\vspace{0.5cm}
\noindent\textbf{关键词:}学术写作;学习档案;检索增强生成;GraphRAG;LoRA/QLoRA;跨端统一 SDK
\clearpage
