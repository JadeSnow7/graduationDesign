\chapter*{摘要}
\addcontentsline{toc}{chapter}{摘要}
本文面向《电磁场与电磁波》课程知识结构复杂、概念抽象与答疑成本高等问题,围绕“知识图谱构建—大语言模型协同问答—系统原型实现—测试评估”展开研究。系统以企业微信内嵌 H5 或网页作为使用入口,采用前端—后端—AI 服务的分层架构,后端提供统一鉴权与权限控制。知识图谱由课程资料抽取、融合与结构化存储,支持概念、公式与定理等知识点的组织与检索;协同问答引擎引入 GraphRAG 检索增强,在图结构索引上下文中调用大语言模型,实现课程概念解释、公式理解与习题求解引导,并通过可控提示模板提升回答的可靠性与可追溯性。原型系统完成了知识库索引构建、问答交互与企业微信/网页集成方案,并给出了功能、准确性与性能的测试与评估思路,为后续规模化部署与教学验证提供基础。

\vspace{0.5cm}
\noindent\textbf{关键词:}知识图谱;大语言模型;检索增强;智能问答;企业微信;电磁场与电磁波
\clearpage
