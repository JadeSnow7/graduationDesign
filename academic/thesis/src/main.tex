% !TEX program = xelatex
\documentclass[UTF8,zihao=-4,oneside,fontset=fandol]{ctexbook}

\usepackage[a4paper,top=2.5cm,bottom=2.5cm,left=3cm,right=2.5cm]{geometry}
\usepackage{setspace}
\usepackage{graphicx}
\usepackage{booktabs}
\usepackage{tabularx}
\usepackage{array}
\usepackage{amsmath,amssymb}
\usepackage{hyperref}
\usepackage{fancyhdr}
\usepackage{caption}
\usepackage{subcaption}
\usepackage{enumitem}
\usepackage{csquotes}
\usepackage[
  backend=biber,
  style=gb7714-2015,
  sorting=none
]{biblatex}
\addbibresource{references.bib}

\setlength{\parindent}{2em}
\setlength{\parskip}{0pt}
\onehalfspacing

\pagestyle{fancy}
\fancyhf{}
\fancyhead[LE,RO]{\thepage}
\fancyhead[LO]{\nouppercase{\leftmark}}
\fancyhead[RE]{\nouppercase{\rightmark}}
\renewcommand{\headrulewidth}{0.4pt}
\setlength{\headheight}{14.5pt}

\newcommand{\HUSTTitle}{基于大模型的以学生为中心的智能教学平台设计与实现}
\newcommand{\HUSTSubtitle}{——以本科生《电磁场与电磁波》和研究生《学术规范与论文写作》为例}
\newcommand{\HUSTAuthor}{张三}
\newcommand{\HUSTStudentId}{2020123456}
\newcommand{\HUSTMajor}{计算机科学与技术}
\newcommand{\HUSTSchool}{计算机科学与技术学院}
\newcommand{\HUSTAdvisor}{李四 教授}
\newcommand{\HUSTDate}{2026年5月}

\newcommand{\makecover}{%
  \thispagestyle{empty}%
  \begin{center}
    \vspace*{1.5cm}
    {\zihao{2}\bfseries 华中科技大学本科毕业论文\par}
    \vspace{1cm}
    {\zihao{-2}\bfseries \HUSTTitle\par}
    \ifx\HUSTSubtitle\empty\else
      \vspace{0.5cm}
      {\zihao{3}\bfseries \HUSTSubtitle\par}
    \fi
  \end{center}
  \vfill
  \begin{center}
    \renewcommand{\arraystretch}{1.5}
    \begin{tabular}{rl}
      学生姓名: & \HUSTAuthor \\
      学\hspace{2em}号: & \HUSTStudentId \\
      专\hspace{2em}业: & \HUSTMajor \\
      学\hspace{2em}院: & \HUSTSchool \\
      指导教师: & \HUSTAdvisor \\
    \end{tabular}
  \end{center}
  \vfill
  \begin{center}
    \HUSTDate
  \end{center}
  \clearpage
}

\begin{document}
\frontmatter
\makecover
\chapter*{原创性声明}
\addcontentsline{toc}{chapter}{原创性声明}
本人声明所呈交的毕业论文是本人在导师指导下进行的研究成果。除文中已经注明引用的内容外,本论文不包含任何他人已经发表或撰写的研究成果。对本文的研究做出重要贡献的个人和集体,均已在文中以明确方式注明。

\vspace{2cm}
\begin{flushright}
作者签名:\underline{\hspace{4cm}}\\[0.5cm]
日期:\underline{\hspace{4cm}}
\end{flushright}

\clearpage
\chapter*{授权使用声明}
\addcontentsline{toc}{chapter}{授权使用声明}
本人同意华中科技大学保存并向国家有关部门或机构送交本论文的复印件和电子版,允许本论文被查阅和借阅。本人授权华中科技大学将本论文的全部或部分内容编入有关数据库进行检索,可以采用影印、缩印或其他复制手段保存和汇编本论文。

\vspace{2cm}
\begin{flushright}
作者签名:\underline{\hspace{4cm}}\\[0.5cm]
日期:\underline{\hspace{4cm}}
\end{flushright}
\clearpage

\chapter*{摘要}
\addcontentsline{toc}{chapter}{摘要}
本文面向高等教育中"课程类型多样、学生差异显著、过程性反馈成本高、生成式模型可信度难以保障"等问题,设计并实现一套以学生为中心的智能教学平台。平台以学习过程数据为核心,围绕作业提交、对话辅导与学习事件形成可追踪的学生档案,并在此基础上提供可控、可追溯、可迭代的智能辅导能力。系统采用前后端分离与服务化架构:前端使用 React + TypeScript + Vite 构建 Web/H5 客户端,可嵌入企业微信 WebView;后端使用 Go + Gin 提供业务 API、JWT 鉴权与 RBAC 权限管理;AI 服务基于 Python + FastAPI 负责编排对话、写作分析、引导式学习、工具调用与检索增强,通过 OpenAI-compatible 接口对接推理服务。异构加速层支持 GPU、NPU 与 FPGA 三种后端:GPU/NPU 用于大模型推理,FPGA 用于 Embedding 服务的低延迟加速与资源解耦,在保持检索质量的前提下显著降低能耗。为降低幻觉并提升可复核性,平台引入 GraphRAG 检索增强,将课程资料构建为图结构索引,生成回答时注入证据片段并按编号引用;对于数值计算、仿真或格式检查等可验证任务,系统通过工具调用将关键结论锚定在可执行结果上。为支持模型定制与持续迭代,平台提供数据规范、LoRA/QLoRA 微调与离线评测脚本,并引入数据蒸馏与 smoke 验证用于训练链路自检。本文以电磁场推导型课程与研究生专业英文写作课程作为示例场景进行验证,展示平台在不同课程中的可迁移性与工程可落地性。

\vspace{0.5cm}
\noindent\textbf{关键词:}智能教学平台;学生档案;GraphRAG;工具调用;引导式学习;异构加速;FPGA
\clearpage

\chapter*{Abstract}
\addcontentsline{toc}{chapter}{Abstract}
This thesis addresses common challenges in higher education, including diverse course types, significant learner differences, the high cost of process-oriented feedback, and the difficulty of guaranteeing reliability in generative models. We design and implement a student-centric intelligent teaching platform powered by large language models (LLMs). The platform organizes learning-process data (submissions, tutoring dialogues, and learning events) into longitudinal student profiles, enabling controllable, traceable, and iterative tutoring. The system follows a service-oriented architecture with a separated front end and back end: the client is built with React, TypeScript, and Vite for web/H5 (embeddable in a WeCom WebView); the backend uses Go and Gin to provide business APIs with JWT-based authentication and RBAC; and the AI service is implemented with Python and FastAPI to orchestrate chat, writing analysis, guided learning, tool calling, and retrieval augmentation, connecting to upstream inference via an OpenAI-compatible API. The heterogeneous acceleration layer supports GPU, NPU (Huawei Ascend), and FPGA (Xilinx Alveo) backends: GPU/NPU for LLM inference, and FPGA for low-latency embedding acceleration and resource decoupling, significantly reducing power consumption while maintaining retrieval quality. To reduce hallucinations and improve auditability, we introduce GraphRAG: course materials are indexed as a graph-structured knowledge base, and responses are grounded on retrieved evidence with explicit citations. For verifiable tasks such as numerical calculation/simulation or format checks, tool calling delegates key steps to executable tools and injects results back into the dialogue. To support model customization and continuous iteration, we provide a data specification, LoRA/QLoRA fine-tuning and offline evaluation scripts, as well as data distillation and smoke validation for pipeline sanity checks. Two representative scenarios---an electromagnetics derivation-intensive course and a graduate academic writing course---are used to demonstrate portability and engineering feasibility.

\vspace{0.5cm}
\noindent\textbf{Keywords:} intelligent teaching platform; student profile; GraphRAG; tool calling; guided learning; heterogeneous acceleration; FPGA
\clearpage

\tableofcontents
\listoffigures
\listoftables

\mainmatter
\chapter{绪论}
\section{研究背景与意义}
《电磁场与电磁波》课程是电子信息类、通信工程等专业的核心基础课程,知识点抽象、公式推导多且逻辑链条长,学生在自学与复习过程中容易出现概念混淆与步骤缺失。传统课堂与常规教学平台在即时答疑、过程性引导与知识结构呈现方面能力有限,教师在作业批改、答疑与学情分析中投入大量重复性工作。近年来,Transformer 架构\cite{vaswani2017}推动大语言模型在自然语言理解与生成方面取得突破,对话式模型(如 ChatGPT)\cite{openai2022}显示出辅助教学与答疑的潜力,但其幻觉与可解释性问题仍需外部知识约束。检索增强生成(RAG)\cite{lewis2020}与知识图谱\cite{hogan2020}提供了结构化知识支撑与可追溯机制,为构建面向课程教学的智能问答系统提供了新的技术路径。基于企业微信的 H5 入口具备便捷触达与组织管理优势,适合承载课程智能教学平台原型。

\section{国内外研究现状}
在教育信息化领域,学习管理系统与学习分析工具已实现课程资源管理、作业统计与学习过程记录,但对推导型课程的概念解释与解题引导支持不足。大语言模型在教育问答、自动反馈等场景中应用快速增长,研究重点从“能回答”转向“可追溯、可验证”,RAG 被广泛用于降低幻觉并提升答案依据\cite{lewis2020}。知识图谱作为结构化语义组织方式,在教材知识组织、概念关联与搜索推荐中具有优势\cite{hogan2020},但与大模型深度协同的课程问答系统仍处于探索阶段。综上,面向《电磁场与电磁波》课程构建知识图谱并与大模型协同的问答引擎,具有明确的研究与应用价值。

\section{研究内容与任务要求}
本文围绕课程知识图谱与协同问答引擎的构建开展研究,主要内容与任务要求如下。

\subsection{课题内容}
\begin{enumerate}[label=(\arabic*)]
  \item 《电磁场与电磁波》课程知识图谱构建。
  \item 大语言模型与知识图谱的协同问答引擎设计。
  \item 系统原型实现与企业微信集成或网页实现。
  \item 系统测试与评估。
\end{enumerate}

\subsection{课题任务要求}
\begin{enumerate}[label=(\arabic*)]
  \item 深入理解大语言模型的基本原理及其应用范式,掌握至少一种主流 LLM 的 API 调用方法。
  \item 掌握知识图谱的构建流程,能够针对《电磁场与电磁波》内容进行知识抽取、融合与存储。
  \item 完成一个包含知识图谱管理、智能问答交互集成功能的完整系统原型。
  \item 系统应能准确回答课程相关的概念、公式、定理等基础问题,并能进行简单的习题求解引导。
  \item 完成毕业设计论文的撰写,论文应结构清晰、论证充分、代码和数据详实。
\end{enumerate}

\section{论文结构}
本文共分三章:第一章介绍研究背景、研究现状以及课题内容与任务要求;第二章给出需求分析、系统架构与关键模块设计实现;第三章总结工作并展望后续改进方向。

\chapter{方法与实现}
\section{需求分析与总体设计}
平台面向管理员、教师/助教与学生三类主要角色,核心需求可归纳为三条主线:其一是\textbf{过程性数据沉淀},能够围绕写作提交、修改与对话辅导记录学习事件,并形成长期可追踪的学生画像;其二是\textbf{写作类型感知的反馈},针对文献综述、课程论文、学位论文与摘要等不同类型提供差异化 rubric 与结构化建议;其三是\textbf{可控与可追溯},在大模型生成建议时提供证据引用与复核入口,降低幻觉带来的教学风险。

非功能需求方面,系统需要具备可扩展与可维护的工程结构(便于迭代模型与课程模块)、清晰的权限边界(避免越权访问与答案泄露)、以及跨终端一致的调用契约(避免 Web/Mobile 接口分叉)。基于上述需求,本文采用前后端分离与服务化架构,将教学业务、AI 能力与检索索引解耦,通过统一鉴权与模块门控策略保证能力可治理。

\section{系统架构}
系统整体采用“客户端—后端业务—AI 服务—检索/存储”的分层结构:客户端提供 Web 与移动端入口;后端提供业务 API、JWT 鉴权、RBAC 权限与课程模块门控;AI 服务负责提示模板、写作分析与对话能力的编排,并通过 OpenAI-compatible 接口调用上游大模型推理服务;GraphRAG 作为可选组件提供课程知识库检索与引用溯源。整体架构如图~\ref{fig:architecture} 所示。

\begin{figure}[htbp]
  \centering
  \fbox{\rule{0pt}{5cm}\rule{10cm}{0pt}}
  \caption{平台总体架构示意(占位)}
  \label{fig:architecture}
\end{figure}

\section{统一契约与跨端共享}
为避免跨端开发中出现“同一业务多套接口/字段”的问题,系统采用 Monorepo 组织代码,并抽取共享包沉淀 types 与统一 SDK。共享 SDK 负责:
(1) 统一请求层(鉴权头、错误归一、超时与重试策略);
(2) 以类型定义约束前后端契约,减少字段不一致导致的运行时错误;
(3) 为 Web/Mobile 提供一致的 API 调用方式,使平台差异主要集中在 UI 与交互层。
该设计降低了多端协作成本,并为后续在不同课程模块间复用能力提供基础。

\section{学生中心数据模型与学习事件}
平台以学生为中心组织数据。在课程层面,写作提交被建模为可追踪的业务对象:包含写作类型、标题与内容、提交时间、AI 分析结果与教师反馈等字段;在过程层面,系统记录关键学习事件(如写作提交、写作分析完成、对话辅导、学习时长心跳等),用于后续聚合形成学生画像。画像不仅包含“分数”,更强调可解释的能力维度(例如结构清晰度、证据使用、学术语气与引用规范),从而支持纵向对比与个性化干预。

\section{引导式学习与薄弱点追踪}
除“写作提交—分析—反馈”流程外,平台提供面向过程性辅导的引导式学习能力(guided learning):系统首先为某一学习主题生成 3--6 步的学习路径(learning path),随后以苏格拉底式提问引导学生逐步完成每一步。例如,在写作课程中,学习主题可围绕 thesis statement、段落结构、证据使用与引用规范等展开,系统会在每轮对话中只提出一个关键问题,并根据学生回答的完整性决定是否进入下一步,从而把复杂能力训练拆解为可管理的阶段。

实现上,AI 服务提供 \texttt{/v1/chat/guided} 端点,使用会话状态(session\_id)维护学习目标、当前步骤与路径结构,并在首轮由模型输出 JSON 路径以便前端渲染进度。为将对话信号沉淀为画像特征,系统在每次对话后对助教回复中的纠错与提示语句做轻量检测,提取与写作相关的薄弱点概念(如“逻辑连接”“引用规范”“论点展开”),记录到会话中并可同步到后端学习档案。结合 GraphRAG 时,系统会把检索到的课程规范与示例片段作为证据注入对话上下文,要求回答标注引用编号并在证据不足时追问或拒答,从而提升引导式建议的可追溯性与可复核性。

\section{写作类型感知的智能分析服务}
写作分析服务以“写作类型 + rubric + 结构化输出”为核心。服务端首先识别或校验写作类型,并选择对应的评估维度与权重;随后调用上游大模型生成反馈,并将输出解析为维度评分、优点与改进建议等结构化字段,便于前端展示与教师复核。与通用润色工具不同,本文更关注“可执行建议”:例如指出段落功能缺失、论证链条不完整、证据不足或引用格式问题,并给出可操作的修改方案。该设计使反馈更贴近课程要求,也更便于后续沉淀高质量标注数据。

\section{GraphRAG 知识库与检索增强生成}
为降低大模型在写作辅导中的幻觉风险,系统引入 GraphRAG 检索增强生成流程\cite{lewis2020}。针对研究生《学术规范与论文写作》课程中常见的“引用格式错误”与“学术不端风险”,本文构建了基于向量检索的课程知识库。构建过程如下:首先,将课程讲义、学校学位论文写作规范及优秀的历年范文进行结构化清洗,并按照“章节—段落”的层级进行切分(Chunking),默认切片大小设定为 1200 字符。其次,采用 `text-embedding-v3` 模型将切分后的文本片段转化为高维度向量(Embedding),并存储于 FAISS 向量数据库中。
当学生在对话中咨询关于引用规范或格式要求的问题时,系统首先将用户查询转化为向量,通过余弦相似度在向量数据库中检索最相关的 $K$ 个规范条款或范文片段。检索到的片段被作为“证据(Evidence)”注入到大模型的上下文提示词(Prompt)中,并强制要求模型仅依据检索到的证据回答,并在回答末尾标注引用来源编号。这一“向量化—检索—注入—生成”的闭环不仅显著降低了模型的幻觉风险,更模拟了真实的学术问题解决过程——即“查阅规范—理解条款—应用执行”,从而在技术实现的底层逻辑上契合了课程的教学目标。

\subsection{学术规范向量知识库构建过程}
针对《学术规范与论文写作》课程中“建议容易泛化、缺少课程依据”的问题,系统将课程规范资料与论文写作指南按统一流程构建为向量知识库。首先,离线 ingestion 支持 `.md/.markdown/.pdf/.txt` 多源文本输入;随后按章节与段落进行切分,并以 `--chunk-chars=1200` 作为默认分块上限,将原始文本转换为可检索的知识片段。接着,系统对片段执行 Embedding 向量化(`api|local|hash|env`,默认模型 `text-embedding-v3`),并将向量写入 `FAISS` 向量存储,同时维护图索引中的节点与邻接关系以支持图扩展检索。

在线推理阶段,查询先经过语义检索与关键词/图扩展召回,再将命中的证据片段注入提示词上下文,约束模型在证据范围内生成并标注引用编号。该“原文资料 $\rightarrow$ chunks $\rightarrow$ embeddings $\rightarrow$ vector store $\rightarrow$ 检索注入生成”的链路,将回答从“语言模型先验”转为“课程证据驱动”,可显著降低《学术规范与论文写作》辅导中的幻觉建议与领域知识缺失问题,并提升教师复核与过程追溯的可行性。

\section{工具调用与可验证能力}
除写作建议外,教学场景中仍存在需要“可验证计算/查询”的任务,例如对字数、结构要素或格式规则进行检查,或在理工类课程中进行数值计算与仿真。为此,AI 服务提供工具调用接口,使模型在需要精确结果时可调用外部工具并将结果回注到对话中,再生成解释性回答。工具调用能力本质上为系统提供了“外部可验证执行器”,用于约束模型的自由生成范围,降低“凭空计算/编造规则”的风险。本文在原型中实现了基础工具集合,并预留面向写作场景的扩展空间(如引用格式校验、结构要素检查等)。

\section{模型后训练与评测管线}
为使模型更贴近课程风格与任务需求,本文实现了面向写作/对话数据的后训练管线:包括数据规范、数据准备脚本、LoRA/QLoRA 微调脚本\cite{hu2021lora,dettmers2023qlora}与离线评测脚本。训练数据以多轮对话 JSONL 表示,并区分 tool/rag/style 等样本类型;评测阶段以固定回归集输出指标与案例,辅助迭代数据与提示策略。受数据规模与时间限制,本文先使用小规模样例数据完成端到端验证:训练脚本可稳定产出 adapter,评测脚本可输出困惑度、格式一致性与拒答准确率等指标,为后续在 Qwen3 8B 上进行 100k 规模训练提供工程基础。

为降低“数据格式不一致导致训练失败”的工程风险,本文在训练前增加了数据蒸馏与冒烟验证步骤:将 chat-style 的训练/评测 JSONL 通过 \texttt{scripts/ai/distill\_data.py} 蒸馏为 prompt/response 格式,并用 \texttt{scripts/ai/train\_smoke.py} 在分钟级输出困惑度等轻量指标,用于验证数据链路与指标输出链路可复现。需要强调的是,smoke 指标仅用于证明训练与评测链路可用,并不代表最终模型效果。

\begin{table}[htbp]
  \centering
  \caption{样例训练链路验证结果(用于证明训练与评测链路可用)}
  \label{tab:smoke-train}
  \begin{tabular}{lccp{6.2cm}}
    \toprule
    指标 & 训练集 & 验证集 & 说明 \\
    \midrule
    样本数 & 3 & 2 & 小规模 JSONL 样例数据,仅用于链路验证 \\
    Token 数 & 68 & 50 & 以分词后 token 计 \\
    困惑度(PPL) & 32.95 & 41.30 & 使用轻量模型完成端到端训练与评测,数值不代表最终效果 \\
    \bottomrule
  \end{tabular}
\end{table}

\subsection{阶段性训练结果同步(2026-02-08)}
在完成训练脚本与评测脚本的端到端连通验证后,项目于 2026-02-08 执行了首次 \texttt{all} 多任务训练评测(小样本)与随机三组回归测试,并将结果同步为统一事实源。指标如表~\ref{tab:stage-train-20260208} 所示。

\begin{table}[htbp]
  \centering
  \caption{阶段性训练结果(2026-02-08,同步批次)}
  \label{tab:stage-train-20260208}
  \begin{tabular}{lp{2.0cm}cccc}
    \toprule
    评测批次 & 样本规模 & key\_point\_coverage & refusal\_accuracy & response\_format & tool\_call\_accuracy \\
    \midrule
    首次 all 训练 & $n=5$ & 0.9167 & 0.8000 & 1.0000 & 0.0000 \\
    随机三组回归均值 & $3 \times n=6$ & 0.7333 & 0.7778 & 0.8333 & 0.0000 \\
    \bottomrule
  \end{tabular}
\end{table}

需要说明的是,上述结果仅用于证明“训练—评测—文档同步”链路可复现,属于阶段性验证数据,不作为本文正式实验结论。后续正式实验将在真实 \texttt{style/tool/rag} 数据闭环后重新训练并报告主结果。

\section{系统原型实现与企业微信集成}
平台前端采用 React + TypeScript 实现 Web 客户端,并提供基于 Expo 的移动端实现;后端采用 Go/Gin 提供课程、写作与学习事件相关 API,并通过 JWT 与 RBAC 实现权限治理;AI 服务基于 FastAPI,实现对话、写作分析、GraphRAG 与工具调用等能力,并对接 OpenAI-compatible 上游推理服务。原型部署层面提供 Docker Compose 配置以便快速启动与验证;对于企业微信等场景,系统预留 OAuth 与组织对接能力,以支持后续在真实教学流程中落地。

\section{系统测试与评估}
系统测试与评估围绕功能正确性、可追溯性与工程稳定性展开。功能测试验证写作提交与分析流程、权限校验、学习事件记录与查询等关键链路;可追溯性测试关注 RAG 模式下的引用一致性与“证据不足时追问/拒答”行为;工程测试关注典型请求的响应时间与服务稳定性。对于模型效果评估,本文采用“固定回归集 + 案例分析”的方式进行离线对比,并预留进一步的用户试用与课堂验证方案,用于在真实课程中评估建议的可采纳性与对学习效果的影响。

\chapter{总结与展望}

\section{工作回顾}
本文针对多课程教学场景中"反馈成本较高、学习过程难以追踪、生成式模型可信度不足"等问题,设计并实现了一套以学生为中心的智能教学平台。下面从四个方面对所做工作进行回顾。

\subsection{系统架构与工程实现}
平台采用前后端分离与服务化架构:前端以 React + TypeScript + Vite 构建 Web/H5 客户端,可嵌入企业微信 WebView 并保持跨端一致体验;后端基于 Go + Gin 提供业务 API,使用 JWT 进行无状态认证,配合 RBAC 模型实现多角色权限管理;AI 服务以 Python + FastAPI 实现,通过 OpenAI-compatible 接口与上游推理服务对接。代码采用 Monorepo 组织,前后端共享类型定义,降低接口不一致风险。在异构加速方面,系统支持 GPU、NPU 与 FPGA 三种后端——GPU/NPU 用于大模型推理,FPGA 用于 Embedding 服务的低延迟加速与资源解耦。

\subsection{可追溯与可验证的智能辅导}
为应对大语言模型的幻觉与可信度问题,平台构建了"证据链 + 计算链"的双重可追溯机制:
\begin{enumerate}[label=(\arabic*)]
  \item \textbf{GraphRAG 检索增强}:将课程讲义与规范构建为图结构索引,采用混合检索与图扩展获取相关证据,回答中以编号形式标注来源,使结论可追溯。
  \item \textbf{工具调用}:对数值计算、仿真与格式检查等任务,通过工具调用获取可执行结果,降低模型"心算失误"与"凭空建议"的风险。
\end{enumerate}

\subsection{过程性辅导与学习追踪}
平台以引导式学习为核心交互方式,将复杂学习主题拆解为可管理的步骤,逐步提问推进,实现"诊断—引导—巩固"的过程性辅导。同时,通过学习事件流与多级学习画像(课程画像/全局画像)将薄弱点、学习时长与完成主题等信息沉淀为可查阅档案,为教师侧学情分析与个性化干预提供数据支撑。

\subsection{训练与评测链路}
为支持模型定制与持续迭代,平台提供完整的后训练工具链:数据规范化确保训练与上线协议一致;数据蒸馏与 smoke 验证作为质量门禁;LoRA/QLoRA 参数高效微调输出可版本化的 adapter;离线回归评测通过固定评测集量化引用一致性、工具调用准确率与结构化输出稳定性等指标。

\section{主要贡献}
本文的主要贡献可归纳为以下五点:
\begin{enumerate}[label=(\arabic*)]
  \item 提出并落地"学生中心数据闭环 + GraphRAG 可追溯 + 工具调用可验证 + 引导式学习"的一体化框架,形成可复用的工程路径。
  \item 设计"通用能力 + 课程专属模块"的模块化策略,以电磁场与研究生专业英文写作为例验证跨课程可迁移性。
  \item 对比 GPU 与 NPU 两种大模型推理后端的性能与能耗,为校园场景下的国产化与绿色部署提供参考。
  \item 引入 FPGA 加速 Embedding 服务,实现检索链路与大模型推理的资源解耦,在保持检索质量的前提下显著降低能耗。
  \item 建立面向持续迭代的训练与回归评测流程,使模型迭代从主观判断转向可量化回归。
\end{enumerate}

\section{不足与改进方向}
尽管本文完成了平台原型实现,仍存在以下局限:

\subsection{当前不足}
\begin{enumerate}[label=(\arabic*)]
  \item \textbf{知识图谱粒度}:当前 GraphRAG 以文档片段为节点,对概念、公式与物理量之间细粒度关系的建模仍有欠缺。
  \item \textbf{评测集规模}:回归集约百条量级,难以覆盖所有边界情况;需进一步扩展并引入人工评测。
  \item \textbf{真实场景验证}:当前验证主要基于模拟数据,尚未在真实课程中大规模部署并收集师生反馈。
  \item \textbf{企业微信深度集成}:OAuth 与消息推送能力处于预留状态,未完成完整的内嵌体验。
  \item \textbf{FPGA 量化损失}:INT8 量化导致 Recall@5 下降约 1.5 个百分点,对高精度场景需进一步优化量化策略。
\end{enumerate}

\subsection{后续工作}
针对上述不足,可从以下方向继续改进:
\begin{enumerate}[label=(\arabic*)]
  \item \textbf{知识抽取与融合}:引入实体关系抽取与图数据库(如 Neo4j),提升知识图谱的细粒度与可维护性。
  \item \textbf{评测体系扩展}:构建千条量级的课程问答评测集,量化准确率与教学效果;引入 A/B 测试框架。
  \item \textbf{检索与上下文优化}:优化检索排序与上下文压缩策略,降低延迟并提升长上下文任务的稳定性。
  \item \textbf{企业微信完整集成}:完成 OAuth 认证与消息推送能力,提升触达体验。
  \item \textbf{多模态能力}:引入图像理解,支持手写作业识别与公式图片解析。
  \item \textbf{学习分析可视化}:为教师提供班级学情看板,展示薄弱点分布与能力发展趋势。
  \item \textbf{异构加速生态完善}:跟进 Ascend 与 Xilinx 软件栈更新,探索 Reranker 的 FPGA 加速与混合精度量化。
\end{enumerate}

\section{结语}
本文完成了一套以学生为中心的智能教学平台原型,在"可追溯、可验证、可迭代"的设计目标下,为将大语言模型可控地引入教育场景提供了工程实践参考。通过引入 GPU、NPU 与 FPGA 三种异构加速后端,平台在性能、能耗与资源利用之间取得了更好的平衡。希望本研究能够为高校智能教学系统的建设提供借鉴,并在后续工作中不断完善与推广。


\appendix
\chapter{附录}
在此放置补充材料,例如证明、额外实验结果或关键代码片段。


\backmatter
\printbibliography[heading=bibintoc]

\end{document}
